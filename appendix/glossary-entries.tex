
% Glossary entries
\newglossaryentry{apig} {
    name=\glslink{apig}{API},
    text=API,
    sort=api,
    description={In informatica con il termine \emph{API}, \emph{Application Programming Interface} (ing. interfaccia di programmazione di un'applicazione), si indica
    un insieme di regole, protocolli e strumenti che consentono a due software o sistemi di interfacciarsi e interagire tra loro in modo standardizzato}
}

\newglossaryentry{umlg} {
    name=\glslink{umlg}{UML},
    text=UML,
    sort=uml,
    description={In ingegneria del software \emph{Unified Modeling Language UML} (ing. linguaggio di modellazione unificato) è un linguaggio di 
    modellazione standardizzato e utilizzato nell'ambito della progettazione di sistemi \emph{software}. L'\emph{UML} fornisce un insieme di simboli grafici, convenzioni e notazioni che consentono di rappresentare
    visivamente i concetti e i comportamenti dei sistemi software}
}

\newglossaryentry{agileg} {
    name=\glslink{agileg}{Agile},
    text=Agile,
    sort=agile,
    description={In ingegneria del software, con il termine \emph{Agile} si indica un approccio metodologico che enfatizza la collaborazione e l'adattabilità nei processi di lavoro.
    L'\emph{Agile} si basa su un insieme di principi descritti nell'\emph{Agile Manifesto}, che promuove valori come la risposta ai cambiamenti dei requisiti o il coinvolgimento attivo del cliente}
}

\newglossaryentry{scrumg} {
    name=\glslink{scrumg}{Scrum},
    text=Scrum,
    sort=scrum,
    description={In ingegneria del software, \emph{Scrum} è un \emph{framework agile} per la gestione del ciclo di sviluppo di un software. È caratterizzato da una struttura organizzativa flessibile e iterativa che promuove la trasparenza, collaborazione e la continua consegna di valore}
}

\newglossaryentry{softwareg} {
    name=\glslink{softwareg}{software},
    text=Software,
    sort=software,
    description={In informatica con il termine \emph{software} si indica l'insieme dei programmi, delle applicazioni, dei dati e delle istruzioni digitali che compongono un sistema informatico}
}

\newglossaryentry{frontendg} {
    name=\glslink{frontendg}{frontend},
    text=frontend,
    sort=frontend,
    description={In informatica con il termine \emph{frontend} ci si riferisce alla parte di un'applicazione o di un sistema software che interagisce direttamente con l'utente. 
    Il \emph{frontend} è responsabile per la presentazione dell'interfaccia utente e per tutti gli aspetti visivi e interattivi dell'applicazione, con cui l'utente può interagire}
}

\newglossaryentry{backendg} {
    name=\glslink{backendg}{backend},
    text=backend,
    sort=backend,
    description={In informatica con il termine \emph{backend} si indica la parte di un'applicazione o di un sistema software che non è visibile all'utente e opera in secondo piano per gestire funzionalità come l'elaborazione dei dati, la gestione del database o l'autenticazione degli utenti}
}

\newglossaryentry{opensourceg} {
    name=\glslink{opensourceg}{open-source},
    text=open-source,
    sort=open-source,
    description={Il termine \emph{open-source} indica un software reso disponibile al pubblico con una licenza che consente l'accesso, la modifica e la sua libera distribuzione}
}

\newglossaryentry{frameworkg} {
    name=\glslink{frameworkg}{framework},
    text=framework,
    sort=framework,
    description={In informatica con il termine \emph{framework} si indica una struttura predefinita e organizzata che fornisce un modello su cui basare e sviluppare applicazioni software}
}

\newglossaryentry{yamlg} {
    name=\glslink{yamlg}{YAML},
    text=YAML,
    sort=yaml,
    description={In informatica con il termine \emph{YAML} (\emph{YAML Ain't Markup Language}) si indica un formato di rappresentazione dei dati progettato per essere facile da leggere e comprendere dall'uomo. È basato su un formato di testo semplice che utilizza l'indentazione per rappresentare la struttura dei dati, rendendolo intuitivo per gli sviluppatori}
}

\newglossaryentry{urig} {
    name=\glslink{urig}{URI},
    text=URI,
    sort=uri,
    description={In informatica con il termine \emph{URI} (\emph{Uniform Resource Identifier}) si indica una sequenza di caratteri che identifica univocamente una specifica risorsa in Internet}
}

\newglossaryentry{domg} {
    name=\glslink{domg}{DOM},
    text=DOM,
    sort=dom,
    description={In informatica con il termine \emph{DOM} (\emph{Document Object Model}) si indica una rappresentazione gerarchica di un documento \emph{HTML}, \emph{XML} o di altre risorse web}
}

\newglossaryentry{mvcg} {
    name=\glslink{mvcg}{MVC},
    text=MVC,
    sort=mvc,
    description={In informatica con il termine \emph{MVC} (\emph{Model-View-Controller}) si indica un design pattern utilizzato nell'ambito dello sviluppo software per organizzare i componenti di un'applicazione}
}

\newglossaryentry{pnpmg} {
    name=\glslink{pnpmg}{pnpm},
    text=pnpm,
    sort=pnpm,
    description={in informatica \emph{pnpm} (\emph{Plug'n'Play Node Package Manager}) indica un sistema di gestione delle dipendenze per progetti \emph{JavaScript} e \emph{Node.js}}
}

\newglossaryentry{dockerg} {
    name=\glslink{dockerg}{docker},
    text=docker,
    sort=docker,
    description={In informatica con il termine \emph{Docker} si indica una piattaforma di virtualizzazione a livello di contenitore che consente di creare, distribuire e gestire applicazioni e servizi in modo isolato}
}

\newglossaryentry{confluenceg} {
    name=\glslink{confluenceg}{Confuence},
    text=Confuence,
    sort=confluence,
    description={\emph{Confluence} è una piattaforma di collaborazione e gestione delle conoscenze sviluppata da \emph{Atlassian}}
}

\newglossaryentry{awsg} {
    name=\glslink{awsg}{AWS},
    text=AWS,
    sort=aws,
    description={\emph{AWS} (\emph{Amazon Web Services}) è una piattaforma di servizi \emph{cloud computing} offerta da \emph{Amazon}}
}

\newglossaryentry{deployg} {
    name=\glslink{deployg}{deploy},
    text=deploy,
    sort=deploy,
    description={In informatica con il termine \emph{deploy} si indica l'atto di mettere in produzione o distribuire un'applicazione \emph{software} in un ambiente operativo}
}

\newglossaryentry{buildg} {
    name=\glslink{buildg}{build},
    text=build,
    sort=build,
    description={In informatica con il termine \emph{build} si indica il processo di compilazione e trasformazione del codice sorgente in un eseguibile}
}

\newglossaryentry{containerg} {
    name=\glslink{containerg}{container},
    text=container,
    sort=container,
    description={In informatica con il termine \emph{container} si indica un'ambiente isolato che contiene il codice e tutte le sue dipendenze, garantendo che l'applicazione funzioni in modo identico su qualsiasi ambiente in cui venga eseguita}
}

\newglossaryentry{endpointg} {
    name=\glslink{endpointg}{endpoint},
    text=endpoint,
    sort=endpoint,
    description={In informatica il termine \emph{endpoint} indica un punto di accesso all'interno di una rete o di un servizio che può essere utilizzato per inviare e ricevere dati}
}

\newglossaryentry{httpg} {
    name=\glslink{httpg}{HTTP},
    text=HTTP,
    sort=http,
    description={In informatica con il termine \emph{HTTP} (\emph{Hypertext Transfer Protocol}) si indica un protocollo di comunicazione che permette lo scambio di dati su Internet}
}

\newglossaryentry{openapig} {
    name=\glslink{openapig}{OpenAPI schema},
    text=OpenAPIs schemas,
    sort=openAPI,
    description={In informatica con il termine \emph{OpenAPI schema} si indica un insieme di standard e convenzioni utilizzato per la documentazione di servizi web \emph{RESTful}}
}

\newglossaryentry{azureadg} {
    name=\glslink{azureadg}{Azure AD},
    text=Azure AD,
    sort=azureactivedirectory,
    description={\emph{Azure AD} (\emph{Azure Active Directory}) è un servizio di gestione degli accessi fornito da \emph{Microsoft}, all'interno della piattaforma \emph{Azure}}
}

\newglossaryentry{pimg} {
    name=\glslink{pimg}{PIM},
    text=PIM,
    sort=pim,
    description={\emph{PIM} (\emph{Product Information Management}) è un sistema utilizzato dalle aziende per raccogliere, gestire e distribuire informazioni dettagliate sui loro prodotti}
}

\newglossaryentry{damg} {
    name=\glslink{damg}{DAM},
    text=DAM,
    sort=dam,
    description={\emph{DAM} (\emph{Digital Asset Management}) è un insieme di tecnologie utilizzate per organizzare, gestire e distribuire risorse digitali come immagini e video}
}


\newglossaryentry{sprintg} {
    name=\glslink{sprintg}{Sprint},
    text=Sprint,
    sort=sprint,
    description={Nel \emph{framework Scrum} lo \emph{Sprint} rappresenta un periodo di tempo limitato durante il quale un team di sviluppo lavora su un insieme di attività concordate}
}

\newglossaryentry{compositionapig} {
    name=\glslink{compositionapig}{Composition API},
    text=Composition API,
    sort=compositionapi,
    description={Nel \emph{framework Vue.js}, la \emph{Composition API} è un paradigma di sviluppo per la creazione di componenti. È un'alternativa alla \emph{Options API} introdotta con la versione 3 di \emph{Vue.js}}
}

\newglossaryentry{rollupjsg} {
    name=\glslink{rollupjsg}{Rollup.js},
    text=Rollup.js,
    sort=rollup.js,
    description={In informatica \emph{Rollup.js} è uno strumento di \emph{bundling JavaScript} che consente di organizzare e raggruppare più moduli in un singolo file}
}

\newglossaryentry{esmg} {
    name=\glslink{esmg}{ESM},
    text=ESM,
    sort=esm,
    description={In informatica \emph{ESM} (\emph{ECMAScript Modules}) rapprensenta uno standard per la gestione dei moduli in \emph{JavaScript}. Sono un'implementazione comune che consente l'importazione
    ed esportazione di moduli con il vantaggio di migliorare la modularità del codice}
}

\newglossaryentry{hotreloadg} {
    name=\glslink{hotreloadg}{hot-reload},
    text=hot-reload,
    sort=hot-reload,
    description={In informatica con il termine \emph{hot-reload} si indica una funzionalità comune adottata nei \emph{framework}, che consente agli sviluppatori di apportare modifiche al codice sorgente
    in tempo reale mentre l'applicazione è in esecuzione, senza doverla riavviare}
}

\newglossaryentry{expressjsg} {
    name=\glslink{expressjsg}{Express.js},
    text=Express.js,
    sort=express.js,
    description={In informatica \emph{Express.js} rappresenta un \emph{framework} per applicazioni web \emph{Node.js} che fornisce un insieme di funzionalità per semplificare la creazione di applicazioni web e \emph{API}}
}


\newglossaryentry{awscodebuildg} {
    name=\glslink{awscodebuildg}{AWS CodeBuild},
    text=AWS CodeBuild,
    sort=awscodebuild,
    description={\emph{AWS CodeBuild} è un servizio di elaborazione gestito da \emph{Amazon} che consente di automatizzare il processo di compilazione e distribuzione del codice sorgente delle applicazioni}
}

\newglossaryentry{awscloudfrontg} {
    name=\glslink{awscloudfrontg}{AWS CloudFront},
    text=AWS CloudFront,
    sort=awscloudfront,
    description={\emph{AWS CloudFront} è un servizio di \emph{Content Delivery Network} (\emph{CDN}) gestito da \emph{Amazon}. Offre un modo semplice per distribuire contenuti web, immagini o altri file multimediali a livello globale}
}

\newglossaryentry{s3g} {
    name=\glslink{s3g}{Bucket S3},
    text=Bucket S3,
    sort=bucket-s3,
    description={In informatica un \emph{bucket S3} rappresenta un contenitore di oggetti all'interno del servizio di archiviazione \emph{Amazon S3}. È uno spazio di archiviazione dove è possibile archiviare qualsiasi tipo di dato, come file multimediali, documenti o dati di applicazioni}
}

\newglossaryentry{lambdag} {
    name=\glslink{lambdag}{lambda},
    text=lambda,
    sort=lambda,
    description={In informatica con il termine \emph{lambda} si indica un servizio di elaborazione \emph{serverless} fornito da \emph{Amazon} che consente agli sviluppatori di eseguire codice senza la necessità di gestire server o infrastrutture sottostanti}
}

\newglossaryentry{serverlessg} {
    name=\glslink{serverlessg}{serverless},
    text=serverless,
    sort=serverless,
    description={In informatica con il termine \emph{serverless} si indica un paradigma di elaborazione \emph{cloud}, utile agli sviluppatori per creare e gestire applicazioni senza la necessità di gestire server o infrastrutture sottostanti}
}

\newglossaryentry{hiddeniframeg} {
    name=\glslink{hiddeniframeg}{hidden iframe},
    text=hidden iframe,
    sort=hidden-iframe,
    description={In informatica con il termine \emph{hidden iframe} si indica un elemento \emph{HTML} che risulta invisibile all'utente e consente di caricare contenuti in modo asincrono}
}

\newglossaryentry{microsoftgraphg} {
    name=\glslink{microsoftgraphg}{Microsoft Graph},
    text=Microsoft Graph,
    sort=microsoft-graph,
    description={\emph{Microsoft Graph} è un servizio di \emph{Microsoft} che consente di accedere ai dati e alle informazioni di \emph{Microsoft 365}, consentendo l'integrazione con le applicazioni}
}

\newglossaryentry{closureg} {
    name=\glslink{closureg}{closure},
    text=closure,
    sort=closure,
    description={In informatica con il termine \emph{closure} si indica una funzione che è in grado di catturare e memorizzare il contesto in cui è stata dichiarata. Questo significa che una funzione \emph{closure} può accedere alle variabili e dati esterni anche dopo che la funzione che li ha dichiarati è terminata}
}

\newglossaryentry{oidcg} {
    name=\glslink{oidcg}{OpenID Connect},
    text=OpenID Connect,
    sort=openid-connect,
    description={In informatica \emph{OIDC} (\emph{OpenID Connect}) rappresenta un protocollo di autenticazione basato su \emph{OAuth 2.0}, progettato per identificare e autenticare utenti in modo sicuro all'interno di applicazioni web}
}


\newglossaryentry{repositoryg} {
    name=\glslink{repositoryg}{repository},
    text=repository,
    sort=repository,
    description={In informatica con il termine \emph{repository} si indica un ambiente di archiviazione centralizzato dedicatato alla gestione e al versionamento del codice sorgente}
}

\newglossaryentry{pythong} {
    name=\glslink{pythong}{Python},
    text=Python,
    sort=python,
    description={In informatica \emph{Python} rappresenta un linguaggio di programmazione ad alto livello, orientato agli oggetti e interpretato}
}


\newglossaryentry{tokenjwtg} {
    name=\glslink{tokenjwtg}{token JWT},
    text=token JWT,
    sort=token-jwt,
    description={In informatica con il termine \emph{token JWT} (\emph{JSON Web Token}) si indica un formato di token standardizzato che consente di rappresentare in modo sicuro informazioni tra due parti come un oggetto \emph{JSON}. Uno dei suoi tipici scopi è quello di verificare l'autenticità di un utente durante l'autenticazione} 
}


\newglossaryentry{x509g} {
    name=\glslink{x509g}{certificato X.509},
    text=certificato X.509,
    sort=certificato-X509,
    description={In informatica con il termine \emph{X.509} si indica uno standard per la gestione dei certificati digitali utilizzato per autenticare l'identità di entità digitali come siti web, server e utenti}
}


\newglossaryentry{spag} {
    name=\glslink{spag}{single page application},
    text=single page applications,
    sort=single page application,
    description={In informatica con il termine \emph{single page application} (\emph{SPA}) si indica un'applicazione web che funziona interamente su una singola pagina web, senza ricaricare la pagina durante l'utilizzo. La \emph{SPA} utilizza tecniche di caricamento asincrono per caricare rapidamente il contenuto della pagina senza causare iterruzioni all'utente} 
}


\newglossaryentry{mixing} {
    name=\glslink{mixing}{mixin},
    text=mixin,
    sort=mixin,
    description={In \emph{Sass} con il termine \emph{mixin} si indica un costrutto che permette di definire un insieme di regole \emph{CSS} che possono essere riutilizzate in più punti all'interno del foglio di stile}
}