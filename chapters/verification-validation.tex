\chapter{Attività di verifica e validazione}
\label{cap:verifica-validazione}

\intro{In questo capitolo saranno  descritti i processi di verifica e validazione del prodotto.
In particolare verranno illustrati i test implementati, i risultati ottenuti e le possibili migliorie future del prodotto.
}

\section{Test di unità}
Durante lo svolgimento del progetto di stage sono stati implementati dei test di unità, in modo da verificare il corretto funzionamento
dei singoli componenti e delle singole funzionalità implementate.\\
Ciascun test di unità è identificato da un codice univoco, che segue la seguente notazione:
\begin{center}
  \textbf{TU-[Codice]}
\end{center}
dove:
\begin{itemize}
  \item \textbf{TU}: definisce che il seguente test è di unità;
  \item \textbf{Codice}: definisce un numero progressivo che identifica il test.
\end{itemize}
Ad ogni test di unità corrisponderà un risultato, che può essere di due tipi:
\begin{itemize}
  \item \textbf{S}: il test è stato superato;
  \item \textbf{NS}: il test non è stato superato.
\end{itemize}

\clearpage
\subsection*{\emph{Views}}

\begin{center}
  \captionof{table}{Tabella del tracciamento dei test di unità views}\label{tab:test-unita-views}
  \begin{longtable}{|c|c|p{0.5\textwidth}|c|}
  \hline
  \textbf{Codice} & \textbf{Oggetto} & \textbf{Descrizione} & \textbf{Stato}\\
  \hline
  TU-1 &LoginView &Verifica del corretto funzionamento dell'autenticazione, con successivo redirect alla pagina principale &S \\
  \hline
  TU-2 &HomeView &Verifica la corretta visualizzazione della pagina e la corretta validazione dei parametri dell'url &S \\
  \hline
  TU-3 &NotFoundView &Verifica la corretta visualizzazione della pagina e il funzionamento del reindirizzamento alla home &S \\
  \hline
\end{longtable}
\end{center}

\subsection*{\emph{Components}}

\begin{center}
  \captionof{table}{Tabella del tracciamento dei test di unità componenti}\label{tab:test-unita-componenti}
  \begin{longtable}{|c|c|p{0.5\textwidth}|c|}
  \hline
  \textbf{Codice} & \textbf{Oggetto} & \textbf{Descrizione} & \textbf{Stato}\\
  \hline
  TU-4 &HeaderNav &Verifica la corretta visualizazione della barra, verificando anche la visualizzazione dei dati dell'utente loggato  &S \\
  \hline
  TU-5 &MainContent &Verifica la corretta visualizzazione dei dati relativi all'api e al funzionamento del try it out di un endpoint &S \\
  \hline
  TU-6 &SideBar &Verifica la corretta visualizzazione della barra laterale, mostrando le API disponibili correttamente &S \\
  \hline
  TU-7 &StartPage &Verifica la corretta visualizzazione della pagina, controllando i parametri nell'url &S \\
  \hline
  TU-8 &DownloadButton &Verifica il corretto funzionamento del download di una singola API &S \\
  \hline
  TU-9 &LogoutButton &Verifica il corretto funzionamento del logout con popup di un utente &S \\
  \hline
  TU-10 &LoginButton &Verifica il corretto funzionamento del login con popup di un utente &S \\
  \hline
  TU-11 &SearchButton &Verifica il corretto funzionamento del bottone di ricerca &S \\
  \hline
  \multicolumn{4}{|c|}{\textbf{Continuazione della tabella~\ref{tab:test-unita-componenti}}} \\
  \hline
  TU-12 &SearchBar &Verifica la corretta visualizzazione della barra di ricerca &S \\
  \hline
  TU-13 &Autocomplete &Verifica la corretta visualizzazione della lista aggiornata in base all'input corrente nella barra di ricerca. Inoltre verifica l'accessibilità del menù a tendina  &S \\
  \hline
  TU-14 &Chip &Verifica la corretta visualizzazione del client-id corrente, verificando inoltre la funzionalità di reset &S \\
  \hline
  TU-15 &Filter &Verifica il corretto funzionamento del filtraggio, in base al filtro selezionato &S \\
  \hline
  TU-16 &OptionList &Verifica la corretta visualizzazione delle API disponibili &S \\
  \hline
  TU-17 &OptionListItem &Verifica il corretto funzionamento del reindirizzamento all'API selezionata &S \\
  \hline
  TU-18 &SnackBar &Verifica la corretta visualizzazione del messaggio di errore corretto &S \\
  \hline
\end{longtable}
\end{center}

\subsection*{\emph{Utils}}


\begin{center}
  \captionof{table}{Tabella del tracciamento dei test di unità utils}\label{tab:test-unita-utils}
  \begin{longtable}{|c|c|p{0.5\textwidth}|c|}
  \hline
  \textbf{Codice} & \textbf{Oggetto} & \textbf{Descrizione} & \textbf{Stato}\\
  \hline
  TU-19 &Auth &Verifica il corretta funzionamento di tutte le funzionalità riguardanti l'autenticazione & S \\
  \hline
  TU-20 &Debounce &Verifica il corretto funzionamento del delay, verificando che un'azione sia eseguita solo dopo un tempo specificato &S \\
  \hline
  TU-21 &endpointsApiCall &Verifica il corretto funzionamento della chiamata GET all'endpoint creato lato server &S \\
  \hline
  TU-22 &getClients &Verifica il corretto funzionamento della chiamata GET all'endpoint creato lato server  &S \\
  \hline
  TU-23 &getResults &Verifica il corretto funzionamento della chiamata POST all'endpoint creato lato server &S \\
  \hline
  TU-24 &MsGraphApiCall &Verifica il corretto funzionamento delle chiamate verso gli endpoint di Microsoft Graph &S \\
  \hline
\end{longtable}
\end{center}

\section{Test di compatibilità cross-browser}
In questa sezione sono introdotti i test di compatibilità cross-browser, che sono stati implementati per assicurare il corretto funzionamento del prodotto finale su tutti i browser più utilizzati.\\
Ciascun test di unità è identificato da un codice univoco, che segue la seguente notazione:
\begin{center}
  \textbf{TB-[Codice]}
\end{center}
dove:
\begin{itemize}
  \item \textbf{TB}: definisce che il seguente test è di compatoiobilità cross-browser;
  \item \textbf{Codice}: definisce un numero progressivo che identifica il test.
\end{itemize}
Ad ogni test di unità corrisponderà un risultato, che può essere di due tipi:
\begin{itemize}
  \item \textbf{S}: il test è stato superato;
  \item \textbf{NS}: il test non è stato superato.
\end{itemize}

\begin{center}
  \captionof{table}{Tabella del tracciamento dei test di compatibilità cross-browser}\label{tab:test-compatibilita-cross-browser}
  \begin{longtable}{|c|c|p{0.5\textwidth}|c|}
  \hline
  \textbf{Codice} & \textbf{Oggetto} & \textbf{Descrizione} & \textbf{Stato}\\
  \hline
  TB-1 &Applicazione &Si verifica la corretta visualizzazione e il corretto funzionamento dell'applicazione sul \textit{browser Microsoft Edge} &S \\
  \hline
  TB-2 &Applicazione  &Si verifica la corretta visualizzazione e il corretto funzionamento dell'applicazione  sul \textit{browser Google Chrome} &S \\
  \hline
  TB-3 &Applicazione &Si verifica la corretta visualizzazione e il corretto funzionamento dell'applicazione  sul \textit{browser Mozilla Firefox} &S \\
  \hline
  TB-4 &Applicazione &Si verifica la corretta visualizzazione e il corretto funzionamento dell'applicazione  sul \textit{browser Safari} &S \\
  \hline
  TB-5 &Applicazione &Si verifica la corretta visualizzazione e il corretto funzionamento dell'applicazione  sul \textit{browser Brave} &S \\
  \hline
  TB-6 &Applicazione &Si verifica la corretta visualizzazione e il corretto funzionamento dell'applicazione  sul \textit{browser Opera} &S \\
  \hline
\end{longtable}
\end{center}
% \clearpage
\section{Documentazione}
Un obiettivo obbligatorio dello stage era quello di produrre una documentazione sul progetto svolto, sia tecnica che funzionale.
La prima delle due è focalizzata sugli aspetti tecnici e implementativi del progetto, andando a rivolgersi principalemnte ai sviluppatori.
Questo tipo di documentazione infatti, include informazioni dettagliate su componenti e tecnologie utilizzate, in modo che una persona che deve iniziare a lavorare sul progetto
può farlo in autonomia.
D'altro canto, la documentazione funzionale è orientata verso gli utenti finali del prodotto, ai clienti o agli stakeholder.
Essa infatti, fornisce una panoramica degli scenari di utilizzo, delle interazioni dell'utente e delle risposte attese del sistema, creando un guida che affronta i problemi più comuni.\\
La validazione della documentazione è stata effettuata tramite un confronto con il tutor aziendale, che ha fornito un feedback sulle parti da migliorare e su quelle da approfondire.\\
Nel dettaglio le documentazioni sono composte dalle seguenti caratteristiche:
\begin{itemize}
  \item \textbf{Documentazione tecnica}: la documentazione in primis espone tutto ciò che è necessario per iniziare a lavorare da subito al progetto, come la locazione della \textit{repository}, i link 
  alle applicazioni deployate nei vari ambienti di sviluppo o indicazioni su come effettuare il \textit{deploy} del progetto.\\
  Successivamente vengono affrontati vari punti come la struttura del progetto, i comandi per avviare il progetto in locale o informazioni ancora più dettagliate come l'aggiunta di un nuovo endpoint all'interno del progetto backend.
  Grazie a tutte queste informazioni, un nuovo sviluppatore può lavorare in completa autonomia sul progetto;
  \item \textbf{Documentazione funzionale}: la documentazione funzionale è stata scritta in modo da essere comprensibile anche a persone che non hanno conoscenze tecniche.
  Essa infatti, fornisce una panoramica generale dell'utilizzo del progetto, fornendo una guida completa su come utilizzare le varie funzionalità che possono
  causare ad un'utente dubbi o incertezze.
\end{itemize}
% si può aggiungere testo 
Infine, entrambe le documentazioni sono disponibili all'interno del \textit{Confluence} aziendale, in modo che sia facilmente accessibile a tutti i membri del team.

\section{Collaudo}
Durante il periodo di stage, sono stati organizzati degli incontri interni con il tutor aziendale a cadenza settimanale, in cui venivano discussi i progressi
raggiunti e le criticità o dubbi riscontrati. Ciò ha permesso di avere un feedback costante sul lavoro svolto, garantendo una maggiore trasparenza.\\
Questi incontri settimanali erano un momento fondamentale per monitorare l'andamento del mio stage e per affrontare eventuali problematiche in modo tempestivo.


\section{Presentazione finale}
Nell'ultima settimana di stage è stata organizzata una presentazione finale, in cui ho illustrato il lavoro svolto e i risultati ottenuti.
La presentazione è stata fatta davanti a tutta l'azienda, in modo che tutti i dipendenti potessero avere una panoramica del progetto.\\
L'esito della presentazione è stato più che positivo e non sono state evidenziate criticità a seguito delle domande poste dai presenti.\\
La presentazione ha contribuito ad una validazione finale del lavoro svolto ad alto livello.

\section{Migliorie future}
Al termine dell'attività di stage, sono state individuate alcune possibili migliorie future, che potrebbero essere implementate in futuro per migliorare
il prodotto finale.\\
In particolare sono state individuate le seguenti:
\begin{itemize}
  \item \textbf{Gestione di openapi più vecchie della 2.0}: attualmente il prodotto supporta solo openapi versione 2.0 e 3.0, ma sarebbe interessante valutare la possibilità di supportare anche versioni precedenti, in modo da rendere il prodotto più flessibile;
  \item \textbf{Valutare cambio di tecnologia a nuxt}: attualmente il prodotto è stato sviluppato con Vue.js e Nest.js, ma sarebbe interessante valutare un cambio di tecnologia verso nuxt.js, che permette di sviluppare applicazioni server-side-rendered, 
  in modo da non avere due applicazioni separate per il frontend e il backend, ma un'unico progetto che comporta una migliore gestione e manutenibilità.
\end{itemize}








