\chapter{Attività di verifica e validazione}
\label{cap:verifica-validazione}

\intro{In questo capitolo saranno  descritti i processi di verifica e validazione del prodotto.
In particolare verranno illustrati i test implementati, i risultati ottenuti e le possibili migliorie future del prodotto.
}

\section{Test di unità}
intro

Ogni test comprende il corretto funzionamento del rendering del componente.

\subsection*{\emph{Views}}

\begin{center}
  \captionof{table}{Tabella del tracciamento dei test di unità views}\label{tab:test-unita-views}
  \begin{longtable}{|c|c|p{0.5\textwidth}|c|}
  \hline
  \textbf{Codice} & \textbf{Oggetto} & \textbf{Descrizione} & \textbf{Stato}\\
  \hline
  TU-1 &LoginView &Verifica del corretto funzionamento dell'autenticazione, con successivo redirect alla pagina principale &S \\
  \hline
  TU-2 &HomeView &text &S \\
  \hline
  TU-3 &NotFoundView &text &S \\
  \hline
\end{longtable}
\end{center}

\subsection*{\emph{Components}}

\begin{center}
  \captionof{table}{Tabella del tracciamento dei test di unità componenti}\label{tab:test-unita-componenti}
  \begin{longtable}{|c|c|p{0.5\textwidth}|c|}
  \hline
  \textbf{Codice} & \textbf{Oggetto} & \textbf{Descrizione} & \textbf{Stato}\\
  \hline
  TU-4 &HeaderNav &text &Superato \\
  \hline
  TU-5 &MainContent &text &Superato \\
  \hline
  TU-6 &SideBar &text &Superato \\
  \hline
  TU-7 &StartPage &text &Superato \\
  \hline
  TU-8 &DownloadButton &text &Superato \\
  \hline
  TU-9 &LogoutButton &text &Superato \\
  \hline
  TU-10 &LoginButton &text &Superato \\
  \hline
  TU-11 &SearchButton &text &Superato \\
  \hline
  TU-12 &SearchBar &text &Superato \\
  \hline
  TU-13 &Autocomplete &text &Superato \\
  \hline
  TU-14 &Chip &text &Superato \\
  \hline
  TU-15 &Filter &text &Superato \\
  \hline
  TU-16 &OptionList &text &Superato \\
  \hline
  TU-17 &OptionListItem &text &Superato \\
  \hline
  TU-18 &SnackBar &text &Superato \\
  \hline
\end{longtable}
\end{center}

\subsection*{\emph{Utils}}


\begin{center}
  \captionof{table}{Tabella del tracciamento dei test di unità utils}\label{tab:test-unita-utils}
  \begin{longtable}{|c|c|p{0.5\textwidth}|c|}
  \hline
  \textbf{Codice} & \textbf{Oggetto} & \textbf{Descrizione} & \textbf{Stato}\\
  \hline
  TU-19 &Auth &Il progetto deve essere accompagnato da documentazione tecnica e funzionale & Superato \\
  \hline
  TU-20 &Debounce &text &Superato \\
  \hline
  TU-21 &endpointsApiCall &text &Superato \\
  \hline
  TU-22 &getClients &text &Superato \\
  \hline
  TU-23 &getResults &text &Superato \\
  \hline
  TU-24 &MsGraphApiCall &text &Superato \\
  \hline
\end{longtable}
\end{center}



\section{Collaudo}
\section{Documentazione}
Un obiettivo obbligatorio dello stage era quello di produrre una documentazione sul progetto svolto, sia tecnica che funzioanale.
La prima delle due è focalizzata sugli aspetti tecnici e implementativi del progetto, andando a rivolgersi principalemnte a sviluppatori.
Questo tipo di documentazione infatti, include informazioni dettagliate su componenti e tecnologie utilizzate, in modo che una persona che deve iniziare a lavorare sul progetto
può farlo in autonomia.
D'altro canto, la documentazione funzionale è orientata verso gli utenti finali del prodotto, ai clienti o agli stakeholder.
Essa infatti, fornisce una panoramica degli scenari di utilizzo, delle interazioni dell'utente e delle risposte attese del sistema, creando un guida che affronta i problemi più comuni.


La validazione della documentazione è stata effettuata tramite un confronto con il tutor aziendale, che ha fornito un feedback sulle parti da migliorare e su quelle da approfondire.
Infine, la documentazione è disponibile sulla product area di confluence, in modo che sia facilmente accessibile a tutti i membri del team.

\section{Migliorie future}
\section{Presentazione finale}
Nell'ultima settimana di stage è stata organizzata una presentazione finale, in cui ho illustrato il lavoro svolto e i risultati ottenuti.
La presentazione è stata fatta davanti a tutta l'azienda, in modo che tutti i dipendenti potessero avere una panoramica del progetto.
L'esito della presentazione è stato più che positivo. [da continuare]








