\chapter{Attività di verifica e validazione}
\label{cap:verifica-validazione}

\intro{In questo capitolo saranno  descritti i processi di verifica e validazione del prodotto.
In particolare verranno illustrati i test implementati, i risultati ottenuti e le possibili migliorie future del prodotto.
}

\section{Test di unità}
intro

Ogni test comprende il corretto funzionamento del rendering del componente.

\subsection*{\emph{Views}}

\begin{center}
  \captionof{table}{Tabella del tracciamento dei test di unità views}\label{tab:test-unita-views}
  \begin{longtable}{|c|c|p{0.5\textwidth}|c|}
  \hline
  \textbf{Codice} & \textbf{Oggetto} & \textbf{Descrizione} & \textbf{Stato}\\
  \hline
  TU-1 &LoginView &Verifica del corretto funzionamento dell'autenticazione, con successivo redirect alla pagina principale &S \\
  \hline
  TU-2 &HomeView &Verifica la corretta visualizzazione della pagina e la corretta validazione dei parametri dell'url &S \\
  \hline
  TU-3 &NotFoundView &Verifica la corretta visualizzazione della pagina e il funzionamento del reindirizzamento alla home &S \\
  \hline
\end{longtable}
\end{center}

\subsection*{\emph{Components}}

\begin{center}
  \captionof{table}{Tabella del tracciamento dei test di unità componenti}\label{tab:test-unita-componenti}
  \begin{longtable}{|c|c|p{0.5\textwidth}|c|}
  \hline
  \textbf{Codice} & \textbf{Oggetto} & \textbf{Descrizione} & \textbf{Stato}\\
  \hline
  TU-4 &HeaderNav &Verifica la corretta visualizazione della barra, verificando anche la visualizzazione dei dati dell'utente loggato  &S \\
  \hline
  TU-5 &MainContent &Verifica la corretta visualizzazione dei dati relativi all'api e al funzionamento del try it out di un endpoint &S \\
  \hline
  TU-6 &SideBar &Verifica la corretta visualizzazione della barra laterale, mostrando le API disponibili correttamente &S \\
  \hline
  TU-7 &StartPage &Verifica la corretta visualizzazione della pagina, controllando i parametri nell'url &S \\
  \hline
  TU-8 &DownloadButton &Verifica il corretto funzionamento del download di una singola API &S \\
  \hline
  TU-9 &LogoutButton &Verifica il corretto funzionamento del logout con popup di un utente &S \\
  \hline
  TU-10 &LoginButton &Verifica il corretto funzionamento del login con popup di un utente &S \\
  \hline
  TU-11 &SearchButton &Verifica il corretto funzionamento del bottone di ricerca &S \\
  \hline
  TU-12 &SearchBar &Verifica la corretta visualizzazione della barra di ricerca &S \\
  \hline
  TU-13 &Autocomplete &Verifica la corretta visualizzazione della lista aggiornata in base all'input corrente nella barra di ricerca. Inoltre verifica l'accessibilità del menù a tendina  &S \\
  \hline
  TU-14 &Chip &Verifica la corretta visualizzazione del client-id corrente, verificando inoltre la funzionalità di reset &S \\
  \hline
  TU-15 &Filter &Verifica il corretto funzionamento del filtraggio, in base al filtro selezionato &S \\
  \hline
  TU-16 &OptionList &Verifica la corretta visualizzazione delle API disponibili &S \\
  \hline
  TU-17 &OptionListItem &Verifica il corretto funzionamento del reindirizzamento all'API selezionata &S \\
  \hline
  TU-18 &SnackBar &Verifica la corretta visualizzazione del messaggio di errore corretto &S \\
  \hline
\end{longtable}
\end{center}

\subsection*{\emph{Utils}}


\begin{center}
  \captionof{table}{Tabella del tracciamento dei test di unità utils}\label{tab:test-unita-utils}
  \begin{longtable}{|c|c|p{0.5\textwidth}|c|}
  \hline
  \textbf{Codice} & \textbf{Oggetto} & \textbf{Descrizione} & \textbf{Stato}\\
  \hline
  TU-19 &Auth &Verifica il corretta funzionamento di tutte le funzionalità riguardanti l'autenticazione & S \\
  \hline
  TU-20 &Debounce &Verifica il corretto funzionamento del delay, verificando che un'azione sia eseguita solo dopo un tempo specificato &S \\
  \hline
  TU-21 &endpointsApiCall &Verifica il corretto funzionamento della chiamata GET all'endpoint creato lato server &S \\
  \hline
  TU-22 &getClients &Verifica il corretto funzionamento della chiamata GET all'endpoint creato lato server  &S \\
  \hline
  TU-23 &getResults &Verifica il corretto funzionamento della chiamata POST all'endpoint creato lato server &S \\
  \hline
  TU-24 &MsGraphApiCall &Verifica il corretto funzionamento delle chiamate verso gli endpoint di Microsoft Graph &S \\
  \hline
\end{longtable}
\end{center}



\section{Collaudo}
\section{Documentazione}
Un obiettivo obbligatorio dello stage era quello di produrre una documentazione sul progetto svolto, sia tecnica che funzioanale.
La prima delle due è focalizzata sugli aspetti tecnici e implementativi del progetto, andando a rivolgersi principalemnte a sviluppatori.
Questo tipo di documentazione infatti, include informazioni dettagliate su componenti e tecnologie utilizzate, in modo che una persona che deve iniziare a lavorare sul progetto
può farlo in autonomia.
D'altro canto, la documentazione funzionale è orientata verso gli utenti finali del prodotto, ai clienti o agli stakeholder.
Essa infatti, fornisce una panoramica degli scenari di utilizzo, delle interazioni dell'utente e delle risposte attese del sistema, creando un guida che affronta i problemi più comuni.\\
La validazione della documentazione è stata effettuata tramite un confronto con il tutor aziendale, che ha fornito un feedback sulle parti da migliorare e su quelle da approfondire.\\
Infine, la documentazione è disponibile sulla product area di confluence, in modo che sia facilmente accessibile a tutti i membri del team.

\section{Migliorie future}
\section{Presentazione finale}
Nell'ultima settimana di stage è stata organizzata una presentazione finale, in cui ho illustrato il lavoro svolto e i risultati ottenuti.
La presentazione è stata fatta davanti a tutta l'azienda, in modo che tutti i dipendenti potessero avere una panoramica del progetto.\\
L'esito della presentazione è stato più che positivo e non sono state evidenziate criticità a seguito delle domande poste dai presenti.\\
La presentazione ha contribuito ad una validazione del lavoro svolto ad alto livello.








