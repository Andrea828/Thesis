\chapter{Analisi dei requisiti}
\label{cap:analisi-requisiti}

\intro{In questo capitolo viene esposta l'analisi dei requisiti effettuata durante lo stage, dove si
 vanno ad illustrare le funzionalità tramite casi d'uso e requisiti identificati, con l'obiettivo di creare un'immagine
 più chiara e definita del sistema.
 }
 
\section{Descrizione dell'applicazione}

Il progetto consiste nel creare un portale che permetta la consultazione di tutte le \textit{API} THRON con la possibilità di provarle in modo semplice e veloce, direttamente dal portale.
Il prodotto verrà utilizzato internamente all'azienda, più precisamente all'interno della \textit{Product Area}, con lo scopo di facilitare attività di sviluppo, di testing e di supporto durante le attività giornaliere aziendali.\\
Il portale è disponibile in tre ambienti di sviluppo: \textit{development}, \textit{quality} e \textit{production}, ognuno dei quali è identificato da un link diverso, che servirà per accedere al portale in quel determinato ambiente.
La differenza principale tra ogni suddivisione è che i \textit{client-id} sono diversi per ogni ambiente, ciò evita che un utente possa utilizzare un \textit{client-id} di produzione involontariamente, che può causare problemi di sicurezza. Inoltre è utile per evitare prove indesiderate, dato che per esempio una chiamata \textit{delete} su un \textit{client-id} di produzione cancellerebbe effettivamente un dato.\\
Ogni \textit{API} è formata da più \textit{endpoint} al suo interno, che possono essere provati singolarmente. Ogni \textit{endpoint} ha un elenco di possibili risposte che può ritornare, che dipende dai parametri inseriti nella chiamata.
Alcune chiamate come le \textit{POST} o \textit{DELETE} avranno dei parametri obbligatori da inserire, mentre altre chiamate come le \textit{GET}, avranno dei parametri opzionali.\\
In caso l'utente voglia provare le diverse chiamate al di fuori del portale, è possibile scaricare le \textit{API} in formato \textit{YAML}, così da poter usare strumenti di terze parti.


\section{Casi d'uso}\label{sec:usecase}
Questa sezione illustra i casi d'uso individuati nel corso dell'analisi dei requisiti del progetto che sono stati definiti utilizzando il linguaggio \glsfirstoccur{\gls{umlg}} (\textit{Unified Modeling Language}).
Ogni caso d'uso offre una panoramica chiara dei diversi attori coinvolti e delle intereazioni che essi intraprendono nel contesto del sistema.\\
Ogni caso d'uso è indentificato da un codice univoco, che segue la seguente notazione:

\begin{center}
    \textbf{UC[Codice-padre].[Codice-figlio]}
  \end{center}

\subsection{Attori}
Gli attori individuati nel sistema sono i seguenti:
\begin{itemize}
    \item \textbf{Utente non autenticato}: è un utente che non è autenticato nel sistema e non può accedere alle funzionalità del portale;
    \item \textbf{Utente autenticato}: è un utente che è autenticato nel sistema e che può accedere alle funzionalità del portale;
    \item \textbf{\textit{Microsoft 365}}: è un servizio di autenticazione di \textit{Microsoft} che permette di autenticarsi tramite \textit{\textit{account}} aziendale.

\end{itemize}

\clearpage

\subsection{Descrizione del sistema}

Di seguito viene illustrato un diagramma riassuntivo (in figura~\ref{fig:usecase-scenario-principale}) che mostra i casi d'uso individuati nel sistema e le relazioni tra di essi.

\begin{figure}[h]
    \centering
    \makebox[\textwidth][c]{%
        \includegraphics[width=1.3\textwidth]{images/usecase/DeveloperPortalHub.jpg}%
    }%
    \caption{Scenario principale}\label{fig:usecase-scenario-principale}
\end{figure}

\clearpage    


% UC 1 - Inserimento ambiente di sviluppo
\begin{usecase}{1}{Inserimento ambiente di sviluppo}\label{uc:inserimento-ambiente-di-sviluppo}
    \usecaseactors{Utente non autenticato}
    \usecasepre{L'utente non è autenticato e non ha ancora avviato l'applicazione web}
    \usecasedesc{L'utente vuole avviare l'applicazione web e deve scegliere in che ambiente di sviluppo farlo}
    \usecasepost{L'utente avvia l'applicazione nell'ambiente di sviluppo scelto}

    \usecasemain{}
        \begin{enumerate}
            \item L'utente seleziona il link del portale a seconda dell'ambiente di sviluppo che vuole avviare (\textit{development}, \textit{quality} e \textit{production}).
        \end{enumerate}

\end{usecase}

% UC 2 - Login
\begin{usecase}{2}{Login}\label{uc:login}
    \usecaseactors{Utente non autenticato, \textit{Microsoft 365}}
    \usecasepre{ L'utente possiede un \textit{account} valido per autenticazione tramite \textit{Microsoft 365} che appartiene al gruppo autorizzato per il login al sistema. Inoltre l'utente non è autenticato e si trova nella pagina di login}
    \usecasedesc{L'utente vuole accedere al sistema e deve inserire le proprie credenziali per accedervi}
    \usecasepost{L'utente è autenticato correttamente e può procedere con l'utilizzo di tutte le funzionalità disponibili all'interno del sistema}

    \usecasemain{}
        \begin{enumerate}
            \item L'utente inserisce la propria e-mail;
            \item L'utente inserisce la propria password;
            \item \textit{Microsoft 365} verifica le credenziali inserite, in base ai permessi configurati.
        \end{enumerate}

    \usecaseext{}
    \begin{enumerate}
        \item Visualizzazione messaggio di errore login UC3.
    \end{enumerate}

    \usecasegen{Come da figura~\ref{fig:uc:login}:}
    \begin{enumerate}
        \item Inserimento e-mail UC2.1;
        \item Inserimento password UC2.2.
    \end{enumerate}

\end{usecase}

\begin{figure}[!ht] 
    \centering 
    \includegraphics[width=\columnwidth, alt={Caso d'uso relativo al login}]{images/usecase/UC2.jpg}
    \caption{UC2 Login}\label{fig:uc:login}
  \end{figure}
  
\newpage

% UC 2.1 - Inserimento e-mail
\begin{usecase}{2.1}{Inserimento e-mail}\label{uc:inserimento-email}
    \usecaseactors{Utente non autenticato}
    \usecasepre{L'utente possiede un \textit{account} valido per autenticazione tramite \textit{Microsoft 365} che appartiene al gruppo autorizzato per il login al sistema. Inoltre l'utente non è autenticato e si trova nella pagina di login}
    \usecasedesc{L'utente deve inserire la propria e-mail per autenticarsi al sistema}
    \usecasepost{L'utente ha inserito la propria e-mail, può quindi procedere a completare il processo di autenticazione}

    \usecasemain{}
        \begin{enumerate}
            \item L'utente inserisce nell'apposito campo la propria e-mail.
        \end{enumerate}

\end{usecase}


% UC 2.2 - Inserimento password
\begin{usecase}{2.2}{Inserimento password}\label{uc:inserimento-password}
    \usecaseactors{Utente non autenticato}
    \usecasepre{L'utente possiede un \textit{account} valido per autenticazione tramite \textit{Microsoft 365} che appartiene al gruppo autorizzato per il login al sistema. Inoltre l'utente non è autenticato e si trova nella pagina di login}
    \usecasedesc{L'utente deve inserire la propria password per autenticarsi al sistema}
    \usecasepost{L'utente ha inserito la propria password e può concludere il processo di autenticazione}

    \usecasemain{}
        \begin{enumerate}
            \item L'utente inserisce nell'apposito campo la propria password.
        \end{enumerate}

\end{usecase}

\clearpage
% UC 3 - Visualizzazione messggio di errore login
\begin{usecase}{3}{Visualizzazione messaggio di errore login}\label{uc:visualizzazione-errore-login}
    \usecaseactors{Utente non autenticato}
    \usecasepre{L'utente ha inserito una tra le due credenziali e-mail o password in modo errato}
    \usecasedesc{L'utente deve inserire delle credenziali corrette per poter effettuare il login correttamente}
    \usecasepost{L'utente ha inserito una tra le due credenziali errate}
    \usecasemain{}
        \begin{enumerate}
            \item L'utente visualizza un messaggio di errore che lo informa che una delle credenziali che ha inserito per autenticarsi al sistema è sbagliata.
        \end{enumerate}

\end{usecase}


% UC 4 - Visualizzazione home page
\begin{usecase}{4}{Visualizzazione \textit{home page}}\label{uc:visualizzasioen-home-page}
    \usecaseactors{Utente autenticato}
    \usecasepre{L'utente è autenticato ed è stato reindirizzato alla pagina principale}
    \usecasedesc{L'utente vuole visualizzare la pagina principale}
    \usecasepost{L'utente ha visualizzato la pagina principale}

    \usecasemain{}
        \begin{enumerate}
            \item L'utente visualizza la pagina principale.
        \end{enumerate}

    \usecasegen{Come da figura~\ref{fig:uc:visualizzazione-home-page}:}
        \begin{enumerate}
            \item Visualizzazione lista \textit{APIs} disponibili UC4.1;
            \item Visualizzazione \textit{client-id} di default UC4.2;
            \item Visualizzazione dettagli utente autenticato UC4.3.
        \end{enumerate}

\end{usecase}

\begin{figure}[!ht] 
    \centering 
    \includegraphics[width=\columnwidth, alt={Caso d'uso relativo alla visualizzazione della homepage}]{images/usecase/UC4.jpg}
    \caption{UC4 Visualizzazione home page}\label{fig:uc:visualizzazione-home-page}
  \end{figure}

\pagebreak

% UC 4.1 - Visualizzazione lista APIs disponibili
\begin{usecase}{4.1}{Visualizzazione lista \textit{APIs} disponibili}\label{uc:visualizzazione-lista-apis-disponibili}
    \usecaseactors{Utente autenticato}
    \usecasepre{L'utente è autenticato ed è stato reindirizzato alla pagina principale}
    \usecasedesc{L'utente vuole visualizzare la lista di \textit{API} disponibili per la consultazione all'interno del sistema}
    \usecasepost{L'utente ha visualizzato la lista di \textit{API} disponibili all'interno del sistema}
    \usecasemain{}
        \begin{enumerate}
            \item L'utente visualizza la lista di \textit{API} disponibili all'interno del sistema.
        \end{enumerate}

\end{usecase}

% UC 4.2 - Visualizzazione client-id di default
\begin{usecase}{4.2}{Visualizzazione \textit{client-id} di default}\label{uc:visualizzazione-client-id-di-default}
    \usecaseactors{Utente autenticato}
    \usecasepre{L'utente è autenticato ed è stato reindirizzato alla pagina principale}
    \usecasedesc{L'utente vuole visualizzare il \textit{client-id} di default impostato nell'ambiente corrente}
    \usecasepost{L'utente ha visualizzato il \textit{client-id} di default impostato nell'ambiente corrente in cui si trova}

    \usecasemain{}
        \begin{enumerate}
            \item L'utente visualizza il \textit{client-id} di default impostato nell'ambiente corrente in cui si trova.
        \end{enumerate}

\end{usecase}

% \clearpage

% % UC 4.3 Visualizzazione dettagli utente autenticato
\begin{usecase}{4.3}{Visualizzazione dettagli utente autenticato}\label{uc:visualizzazione-dettagli-utente-autenticato}
    \usecaseactors{Utente autenticato}
    \usecasepre{L'utente è autenticato ed è stato reindirizzato alla pagina principale}
    \usecasedesc{L'utente vuole visualizzare i propri dati personali, ovvero del proprio utente autenticato}
    \usecasepost{L'utente visualizza i dati personali del proprio \textit{account} autenticato nel sistema}

    \usecasemain{}
        \begin{enumerate}
            \item L'utente visualizza le informazioni personali del proprio \textit{account} autenticato nel sistema.
        \end{enumerate}

\end{usecase}

% UC 5 Visualizzazione dettagli singola API
\begin{usecase}{5}{Visualizzazione dettagli singola \textit{API}}\label{uc:visualizzazione-dettagli-singola-api}
    \usecaseactors{Utente autenticato}
    \usecasepre{L'utente è autenticato e si trova nella pagina principale}
    \usecasedesc{L'utente vuole visualizzare la pagina di dettaglio di una singola \textit{API}}
    \usecasepost{L'utente ha visualizzato la pagina di dettaglio di una singola \textit{API} tra quelle presenti nel sistema}

    \usecasemain{}
        \begin{enumerate}
            \item L'utente clicca su una delle \textit{API} presenti nella lista di \textit{API} disponibili all'interno del sistema.
            \item L'utente visualizza la pagina di dettaglio della singola \textit{API} selezionata.
        \end{enumerate}

    \usecasegen{Come da figura~\ref{fig:uc:visualizzazione-dettaglio-singola-api}:}
        \begin{enumerate}
            \item Visualizzazione lista \textit{endpoint} disponibili UC5.1.
        \end{enumerate}

\end{usecase}

\begin{figure}[!ht] 
    \centering 
    \includegraphics[width=\columnwidth, alt={Caso d'uso relativo alla visualizzazione del dettaglio di una singola API}]{images/usecase/UC5.jpg}
    \caption{UC5 Visualizzazione dettaglio singola API}\label{fig:uc:visualizzazione-dettaglio-singola-api}
  \end{figure}

  \pagebreak

% UC 5.1 Visualizzazione lista endpoint disponibili
\begin{usecase}{5.1}{Visualizzazione lista \textit{endpoint} disponibili}\label{uc:visualizzazione-lista-endpoint-disponibili}
    \usecaseactors{Utente autenticato}
    \usecasepre{L'utente è autenticato e sta visualizzando la pagina di dettaglio di una singola \textit{API}}
    \usecasedesc{L'utente vuole visualizzare la lista completa di \textit{endpoint} disponibili per l'\textit{API}}
    \usecasepost{L'utente visualizza la lista completa di \textit{endpoint} disponibili per l'\textit{API}}

    \usecasemain{}
        \begin{enumerate}
            \item L'utente visualizza la lista completa di \textit{endpoint} disponibili per l'\textit{API} che ha selezionato.
        \end{enumerate}

    \usecasegen{Come da figura~\ref{fig:uc:visualizzazione-lista-endpoint-disponibili}:}
        \begin{enumerate}
            \item Visualizzazione dettaglio singolo \textit{endpoint} UC5.1.1.
        \end{enumerate}

\end{usecase}

\begin{figure}[!ht] 
    \centering 
    \includegraphics[width=\columnwidth, alt={Caso d'uso relativo al visualizzazione della lista di endpoint disponibili}]{images/usecase/UC5.1.jpg}
    \caption{UC5.1 Visualizzazione lista endpoint disponibili}\label{fig:uc:visualizzazione-lista-endpoint-disponibili}
  \end{figure}
  \pagebreak


% UC 5.1.1 Visualizzazione dettaglio singolo endpoint
\begin{usecase}{5.1.1}{Visualizzazione dettaglio singolo \textit{endpoint}}\label{uc:visualizzazione-dettaglio-singolo-endpoint}
    \usecaseactors{Utente autenticato}
    \usecasepre{L'utente è autenticato, sta visualizzando la pagina di dettaglio di una singola \textit{API} contenente la lista di \textit{endpoint} disponibili}
    \usecasedesc{L'utente vuole visualizzare la pagina di dettaglio di un singolo \textit{endpoint}}
    \usecasepost{L'utente visualizza la pagina di dettaglio di un singolo \textit{endpoint}}

    \usecasemain{}
        \begin{enumerate}
            \item L'utente clicca sullo specifico \textit{endpoint} che vuole visualizzare;
            \item L'utente visualizza i dettagli disponibili per l'\textit{endpoint} selezionato.
        \end{enumerate}

    \usecaseext{}
        \begin{enumerate}
            \item Visualizzazione lista di possibili risposte \textit{endpoint} UC6.
        \end{enumerate}

\end{usecase}


% UC 6 Visualizzazione lista di possibili risposte endpoint
\begin{usecase}{6}{Visualizzazione lista di possibili risposte \textit{endpoint}}\label{uc:visualizzazione-risposte-endpoint}
    \usecaseactors{Utente autenticato}
    \usecasepre{L'utente è autenticato, sta visualizzando la sezione di dettaglio di un singolo \textit{endpoint}}
    \usecasedesc{L'utente vuole visualizzare la lista dei possibili risultati possibili per l'\textit{endpoint} selezionato}
    \usecasepost{L'utente visualizza la lista delle possibili risposte disponibili per l'\textit{endpoint} che ha selezionato}

    \usecasemain{}
        \begin{enumerate}
            \item L'utente visualizza i dettagli riguardanti le possibili risposte che l'\textit{endpoint} selezionato può ritornare.
        \end{enumerate}

\end{usecase}


% UC 7 Download singola API
\begin{usecase}{7}{Download singola \textit{API}}\label{uc:download-singola-api}
    \usecaseactors{Utente autenticato}
    \usecasepre{L'utente è autenticato ed ha selezionato una \textit{API} dalla lista di \textit{API} disponibili nel sistema}
    \usecasedesc{L'utente vuole poter scaricare in formato \textit{YAML} un \textit{API} dal portale}
    \usecasepost{L'utente ha scaricato in formato \textit{YAML} un \textit{API} dal portale}

    \usecasemain{}
        \begin{enumerate}
            \item L'utente scarica il formato \textit{YAML} di un \textit{API} dal portale, tra quelle disponibili;
            \item Una nuova pagina si apre e il download dell'\textit{API} viene eseguito;
            \item Il nome del file è già impostato con il nome dell'\textit{API} scaricata.
        \end{enumerate}

\end{usecase}


% UC 8 Try it out endpoint
\begin{usecase}{8}{\textit{Try it out endpoint}}\label{uc:try-it-out-endpoint}
    \usecaseactors{Utente autenticato}
    \usecasepre{L'utente è autenticato, sta visualizzando i dettagli di un singolo \textit{endpoint} di un \textit{API} disponibile nel sistema}
    \usecasedesc{L'utente vuole poter provare l'\textit{endpoint} selezionato}
    \usecasepost{L'utente ha provato l'\textit{endpoint} selezionato}
    \usecasemain{}
        \begin{enumerate}
            \item L'utente ha selezionato una determinata \textit{API};
            \item L'utente ha selezionato un determinato \textit{endpoint} di quell'\textit{API};
            \item L'utente ha cliccato sul pulsante per provare l'\textit{endpoint} selezionato.
        \end{enumerate}

    \usecaseext{}
        \begin{enumerate}
            \item Visualizzazione errore nella prova dell'\textit{endpoint} selezionato UC9;
            \item Visualizzazione risposta \textit{endpoint} UC10.
        \end{enumerate}

    \usecasegen{Come da figura~\ref{fig:uc:try-it-out-endpoint}:}
        \begin{enumerate}
            \item Inserimento parametri per \textit{try it out endpoint} UC8.1;
            \item Definire campi aggiuntivi per \textit{try it out endpoint} UC8.2.
        \end{enumerate}

\end{usecase}

\begin{figure}[!ht] 
    \centering 
    \includegraphics[width=\columnwidth, alt={Caso d'uso relativo alla prova di un endpoint}]{images/usecase/UC8.jpg}
    \caption{UC8 Try it out endpoint}\label{fig:uc:try-it-out-endpoint}
\end{figure}

% \\

% UC 8.1 Inserimento parametri necessari per la prova dell'endpoint     
\begin{usecase}{8.1}{Inserimento parametri per \textit{try it out endpoint}}\label{uc:inserimento-parametri-try-it-out-endpoint}
    \usecaseactors{Utente autenticato}
    \usecasepre{L'utente è autenticato, sta visualizzando la schermata di \textit{try it out} nella sezione di inserimento dei parametri}
    \usecasedesc{L'utente vuole poter inserire i parametri necessari per la prova dell'\textit{endpoint}}
    \usecasepost{L'utente ha inserito i parametri necessari alla chiamata verso l'\textit{endpoint}}
    \usecasemain{}
        \begin{enumerate}
            \item L'utente inserisce i parametri richiesti per la chiamata verso l'\textit{endpoint} selezionato.
        \end{enumerate}

\end{usecase}


% UC 8.2 Definire campi aggiuntivi per try it out endpoint
\begin{usecase}{8.2}{Definire campi aggiuntivi per \textit{try it out endpoint}}\label{uc:definire-campi-try-it-out-endpoint}
    \usecaseactors{Utente autenticato}
    \usecasepre{L'utente è autenticato, sta visualizzando la schermata di \textit{try it out} nella sezione dei parametri aggiuntivi}
    \usecasedesc{L'utente vuole poter definire dei campi aggiuntivi ai parametri già esistenti, per poi andare a provare la chiamata verso l'\textit{endpoint}}
    \usecasepost{L'utente ha creato dei parametri aggiuntivi per la chiamata verso l'\textit{endpoint}}

    \usecasemain{}
        \begin{enumerate}
            \item L'utente definisce dei parametri da aggiungere a quelli già esistenti.
        \end{enumerate}

\end{usecase}

% UC 9 Visualizzazione errore nella prova dell'endpoint selezionato
\begin{usecase}{9}{Visualizzazione errore nella prova dell'\textit{endpoint} selezionato}\label{uc:visualizzazione-errore-prova-endpoint-selezionato}
    \usecaseactors{Utente autenticato}
    \usecasepre{L'utente è autenticato, è nella sezione di prova di un \textit{endpoint} ed ha inserito dei parametri errati}
    \usecasedesc{L'utente deve inserire dei parametri corretti per poter provare l'\textit{endpoint} senza riscontrare errori} 
    \usecasepost{L'utente ha inserito dei parametri parzialmente o in modo scorretto e non può procedere con l'esecuzione della chiamata}

    \usecasemain{}
        \begin{enumerate}
            \item L'utente visualizza un messaggio di errore che lo avvisa che i parametri inseriti non sono corretti, o risultano incompleti.
        \end{enumerate}

\end{usecase}


% UC 10 Visualizzazione risposta endpoint
\begin{usecase}{10}{Visualizzazione risposta \textit{endpoint}}\label{uc:visualizzazione-risposta-endpoint}
    \usecaseactors{Utente autenticato}
    \usecasepre{L'utente è autenticato, è nella sezione di prova di un \textit{endpoint} ed ha inserito dei parametri corretti}
    \usecasedesc{L'utente vuole visualizzare la risposta dell'\textit{endpoint}}
    \usecasepost{L'utente visualizza la risposta adeguata alla chiamata verso l'\textit{endpoint}, in base ai parametri inseriti}

    \usecasemain{}
        \begin{enumerate}
            \item L'utente visualizza la risposta dell'\textit{endpoint}, uno tra quelle possibili contenute nella descrizione dell'\textit{endpoint}. La risposta varia in base ai parametri inseriti.
        \end{enumerate}

\end{usecase}

\clearpage

% UC 11 Ricerca per client-id
\begin{usecase}{11}{Ricerca per \textit{client-id}}\label{uc:ricerca-client-id}
    \usecaseactors{Utente autenticato}
    \usecasepre{L'utente è autenticato e sta navigando all'interno del sistema} 
    \usecasedesc{L'utente vuole poter ricercare un \textit{client-id} all'interno del sistema}
    \usecasepost{L'utente effettua una ricerca per \textit{client-id} e il sistema ha restituito i risultati della ricerca}

    \usecasemain{}
        \begin{enumerate}
            \item L'utente clicca il bottone per effettuare la ricerca per \textit{client-id};
            \item L'utente inserisce nell'apposito campo il \textit{client-id} che vuole cercare all'interno del sistema;
            \item Il sistema ricerca il \textit{client-id} inserito e restituisce i risultati della ricerca;
            \item Il sistema aggiunge l'ultima ricerca effettuata alla cronologia delle ultime ricerche.
        \end{enumerate}

    \usecaseext{}
        \begin{enumerate}
            \item Visualizzazione messaggio di errore di ricerca UC 12.
        \end{enumerate}

\end{usecase}


% % UC 12 Visualizzazione messaggio di errore di ricerca
\begin{usecase}{12}{Visualizzazione messaggio di errore di ricerca}\label{uc:visualizzazione-errore-ricerca}
    \usecaseactors{Utente autenticato}
    \usecasepre{L'utente è autenticato e ha cercato un \textit{client-id} o un \textit{API} inesistente nel sistema}
    \usecasedesc{L'utente deve inserire un \textit{client-id} o un \textit{API} esistente nel sistema per poter effettuare la ricerca. Inoltre il \textit{client-id} deve esistere per l'ambiente selezionato}
    \usecasepost{L'utente ha inserito un \textit{client-id} o un \textit{API} inesistente nel sistema e visualizza un messaggio di errore}

    \usecasemain{}
        \begin{enumerate}
            \item L'utente visualizza un messaggio di errore che lo informa che la ricerca effettuata non ha prodotto nessun risultato, indicando che ciò è dovuto al fatto
            che il \textit{client-id} o l'\textit{API} cercata non è presente nel sistema, oppure non esiste quel \textit{client-id} per l'ambiente selezionato.
        \end{enumerate}

\end{usecase}

\clearpage
% UC 13 Ricerca per API
\begin{usecase}{13}{Ricerca per \textit{API}}\label{uc:ricerca-api}
    \usecaseactors{Utente autenticato}
    \usecasepre{L'utente è autenticato e sta navigando all'interno del sistema}
    \usecasedesc{L'utente vuole poter ricercare un \textit{API} all'interno del sistema}
    \usecasepost{L'utente effettua una ricerca per \textit{API} e il sistema ha restituito i risultati della ricerca}

    \usecasemain{}
        \begin{enumerate}
            \item L'utente clicca il bottone per effettuare la ricerca per \textit{API};
            \item L'utente inserisce nell'apposito campo l'\textit{API} che vuole cercare all'interno del sistema;
            \item Il sistema ricerca l'\textit{API} inserita e restituisce i risultati della ricerca;
            \item Il sistema aggiunge l'ultima ricerca effettuata alla cronologia delle ultime ricerche.
        \end{enumerate}

    \usecaseext{}
        \begin{enumerate}
            \item Visualizzazione messaggio di errore di ricerca UC12.
        \end{enumerate}

\end{usecase}


% UC 14 Reset client-id corrente
\begin{usecase}{14}{Reset \textit{client-id} corrente}\label{uc:reset-client-id}
    \usecaseactors{Utente autenticato}
    \usecasepre{L'utente è autenticato, sta navigando all'interno del sistema e ha già selezionato un \textit{client-id} che non è quello di default} 
    \usecasedesc{L'utente vuole poter resettare il \textit{client-id} corrente e tornare al \textit{client-id} di default per l'ambiente selezionato}
    \usecasepost{L'utente ha resettato il \textit{client-id} corrente e ora visualizza il \textit{client-id} di default per l'ambiente selezionato}

    \usecasemain{}
        \begin{enumerate}
            \item L'utente clicca il bottone per resettare il \textit{client-id} corrente;
            \item Il sistema resetta il \textit{client-id} corrente ed imposta il \textit{client-id} al valore di default a seconda dell'ambiente selezionato.
        \end{enumerate}
    \clearpage
    \usecaseext{}
        \begin{enumerate}
            \item Visualizzazione messaggio di errore reset UC15.
        \end{enumerate}

\end{usecase}


% UC 15 Visualizzazione messaggio di errore reset
\begin{usecase}{15}{Visualizzazione messaggio di errore reset}\label{uc:visualizzazione-errore-reset}
    \usecaseactors{Utente autenticato}
    \usecasepre{L'utente è autenticato e ha provato a resettare il \textit{client-id} corrente che è quello di default per l'ambiente selezionato}
    \usecasedesc{L'utente deve aver selezionato un \textit{client-id} diverso da quello di default, per poterlo resettare}
    \usecasepost{L'utente ha provato a resettare il \textit{client-id} corrente, ma risulta essere quello di default}

    \usecasemain{}
        \begin{enumerate}
            \item L'utente visualizza un messaggio di errore che lo informa che il \textit{client-id} corrente è quello di default per l'ambiente selezionato e non può essere resettato.
        \end{enumerate}

\end{usecase}


% UC 16 Logout
\begin{usecase}{16}{Logout}\label{uc:}
    \usecaseactors{Utente autenticato, \textit{Microsoft 365}}
    \usecasepre{L'utente è autenticato e vuole uscire dalla sessione corrente}
    \usecasedesc{L'utente vuole effettuare il logout dal sistema}
    \usecasepost{L'utente effettua il logout dal sistema terminando la sessione corrente, non è più autenticato e viene reindirizzato alla pagina di login}

    \usecasemain{}
        \begin{enumerate}
            \item L'utente clicca il bottone per effettuare il logout;
            \item \textit{Microsoft 365} effettua il logout dell'utente terminando la sessione.
        \end{enumerate}

\end{usecase}

\clearpage

\section{Tracciamento dei requisiti}

In questa sezione vengono riportati i requisiti individuati durante il progetto di stage.
Questo capitolo si sofferma in particolare sulla classificazione dei requisiti in tre categorie principali:
\begin{itemize}
    \item \textbf{Requisiti funzionali}: delineano le funzionalità che il sistema deve offrire. Essi delineano le azioni specifiche che il sistema deve eseguire, le risposte attese a determinati input e le dinamiche generali delle operazioni;
    \item\textbf{Requisiti qualitativi}: definiscono gli aspetti legati alla qualità, all'usabilità e alle prestazioni del sistema;
    \item \textbf{Requisiti di vincolo}: delineano le restrizioni e i parametri che il sistema deve rispettare durante lo sviluppo e l'implementazione. 
\end{itemize}

Inoltre viene fatta una classificazione dei requisiti in base alla priorità assegnata a ciascun requisito.

\subsubsection{Notazione}
Ciascun requisito è identificato da un codice univoco, che segue la seguente notazione:
\begin{center}
    \textbf{R[Priorità][Tipo]-[Codice]}
\end{center}
  dove:
  \begin{itemize}
  \item \textbf{Priorità} indica il livello di priorità assegnato: obbligatorio (\textbf{O}), desiderabile (\textbf{D}) e opzionale (\textbf{Z});
  \item \textbf{Tipo} indica il tipo di requisito: funzionale (\textbf{F}) , qualititativo (\textbf{Q}) e di vincolo (\textbf{V});
  \item \textbf{Codice} indica il codice identificativo del requisito.
  \end{itemize}

Nelle tabelle~\ref{tab:requisiti-funzionali},~\ref{tab:requisiti-qualitativi} e~\ref{tab:requisiti-vincolo} sono riassunti i requisiti tramite una breve descrizione accompagnata dalle fonti da cui è stato individuato il requisito per
facilitarne la tracciabilità. 

\pagebreak

\subsubsection{Requisiti funzionali}

\begin{center}
\captionof{table}{Tabella del tracciamento dei requisiti funzionali}\label{tab:requisiti-funzionali}
\begin{longtable}{|c|p{0.6\textwidth}|c|}
\hline
\textbf{Requisito} & \textbf{Descrizione} & \textbf{Fonte}\\
\hline
RFO-1 &L'utente deve scegliere l'ambiente di sviluppo & UC1 \\
\hline
RFO-2 &Il sistema deve permettere l'autenticazione ad un utente con un \textit{account} valido & UC2 \\
\hline
RFO-2.1 & L'utente deve inserire la propria mail & UC2.1 \\
\hline
% \textit{username}
RFO-2.2 & L'utente deve inserire la propria password & UC2.2 \\
\hline
RFO-3 &Il sistema deve avvisare l'utente tramite un messaggio di errore che le credenziali inserite nel login sono errate & UC3 \\
\hline
RFO-4 &Il sistema deve reindirizzare l'utente alla pagina principale, dopo che il login è andato a buon fine & UC4 \\
\hline
RFO-4.1 &L'utente deve poter visualizzare la lista di \textit{API} disponibili nel sistema & UC4.1 \\
\hline
RFO-4.2 &L'utente deve poter visualizzare la lista di \textit{client-id} di default impostata nell'ambiente di sviluppo in cui si trova & UC4.2 \\
\hline
RFO-4.3 &L'utente deve poter visualizzare i dettagli relativi al suo \textit{account} & UC4.3 \\
\hline
RFO-5 &L'utente deve poter visualizzare i dettagli relativi ad una singola \textit{API}  & UC5 \\
\hline
RFO-5.1 &L'utente deve poter visualizzare la lista di \textit{endpoint} disponibili all'interno del sistema & UC5.1 \\
\hline
RFO-5.1.1 &L'utente deve poter visualizzare i dettagli relativi ad un singolo \textit{endpoint} di una determinata \textit{API} & UC5.1.1 \\
\hline
RFO-6 &L'utente deve poter visualizzare l'elenco delle possibili risposte per il determinato \textit{endpoint} selezionato & UC6 \\
\hline
RFO-7 &Il sistema deve permettere il download all'utente di una singola \textit{API} & UC7 \\
\hline
RFO-8 &Il sistema deve permettere il \textit{try it out} di un singolo \textit{endpoint} all'utente & UC8 \\
\hline
RFO-8.1 &Il sistema deve permettere l'inserimento dei parametri necessari per il \textit{try it out} di un \textit{endpoint} & UC8.1 \\
\hline
RFD-8.2 &Il sistema deve permettere la possibilità di definire dei campi aggiuntivi per il \textit{try it out} di un \textit{endpoint} & UC8.2 \\
\hline\pagebreak\hline
\multicolumn{3}{|c|}{\textbf{Continuazione della tabella~\ref{tab:requisiti-funzionali}}} \\
\hline
RFO-9 &Il sistema deve avvisare l'utente tramite un messaggio di errore che i parametri inseriti non sono corretti o incompleti & UC9 \\
\hline
RFO-10 &L'utente deve poter visualizzare la risposta dell'\textit{endpoint} che ha provato & UC10 \\
\hline
RFO-11 &Il sistema deve permettere all'utente di poter effettuare una ricerca per \textit{client-id} & UC11 \\
\hline
RFO-12 &Il sistema deve avvisare l'utente tramite un messaggio di errore che la ricerca effettuata non ha portato a risultati presenti nel sistema & UC12 \\
\hline
RFO-13 &Il sistema deve permettere all'utente di poter effettuare una ricerca per \textit{API} & UC13 \\
\hline
RFD-14 &Il sistema deve permettere il reset del \textit{client-id} corrente & UC14 \\
\hline
RFO-15 &Il sistema deve avvisare l'utente tramite un messaggio di errore che il \textit{client-id} è già di default & UC15 \\
\hline
RFO-16 &Il sistema deve permettere il logout all'utente, uscendo dalla sessione  & UC16 \\
\hline
\end{longtable}
\end{center}

\subsubsection{Requisiti qualitativi}

\begin{center}
\captionof{table}{Tabella del tracciamento dei requisiti qualitativi}\label{tab:requisiti-qualitativi}
\begin{longtable}{|c|p{0.6\textwidth}|c|}
\hline
\textbf{Requisito} & \textbf{Descrizione} & \textbf{Fonte}\\
\hline
RQO-1 &Il progetto deve essere accompagnato da documentazione tecnica e funzionale & Interno \\
\hline
RQO-2 &La parte frontend del progetto deve essere coperta da test di unità & Interno \\
\hline
RQZ-3 &L'applicazione web deve avere un'interfaccia responsive & Interno \\
\hline
RQD-5 &L'applicativo deve essere accessibile utilizzando i principali browser & Interno \\
\hline
\end{longtable}
\end{center}

\pagebreak
\subsubsection{Requisiti di vincolo}

\begin{center}
\captionof{table}{Tabella del tracciamento dei requisiti di vincolo}\label{tab:requisiti-vincolo}
\begin{longtable}{|c|p{0.6\textwidth}|c|}
\hline
\textbf{Requisito} & \textbf{Descrizione} & \textbf{Fonte}\\
\hline
RVO-1 & L'applicazione deve essere sviluppata utilizzando il \textit{framework Vue.js} 3, usando \textit{TypeScript} come linguaggio di programmazione & UC1 \\
\hline
RVO-2 & I componenti devono essere scritti utilizzando le \glsfirstoccur{\gls{compositionapig}} di \textit{Vue.js 3} & UC1 \\
\hline
RVD-3 & I componenti di base devono essere implementati utilizzando la libreria \textit{THRON Components} & UC2 \\
\hline
RVZ-4 & L'interfaccia dell'applicazione deve seguire il \textit{design system} THRON & UC2 \\
\hline
\end{longtable}
\end{center}