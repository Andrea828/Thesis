\chapter{Descrizione del progetto di stage}\label{cap:descrizione-stage}

\intro{Il seguente capitolo vuole introdurre brevemente il progetto affrontato durante lo stage, evidenziando gli obiettivi 
prefissati e i possibili rischi che si potrebbero incontrare.
}

\section{Introduzione al progetto}\label{sec:introduzione-progetto}
Un punto cardine dell'architettura della piattaforma THRON è la suddivisione in micro servizi, che agevolano la manutenibilità e semplificano le operazioni di sviluppo.
Allo stesso tempo rendono però più complessa la consultazione delle \glsfirstoccur{\gls{apig}} esposte e aumentano la quantità di comunicazione necessaria per mantenere tutti i reparti allineati.
Per questo, tra le necessità che stanno avendo le aree di Prodotto e \textit{Revenue} (\textit{Sales}, \textit{Marketing} e \textit{Customer Services}) in THRON, è emersa l'esigenza di avere un unico punto
da cui sia possibile consultare in maniera intuitiva, tutte le interfacce delle \textit{API} che metta a disposizione l'area di Prodotto.\\
Il progetto di stage è consistito nello sviluppo di un portale in \textit{Vue.js} per favorire la consultazione di tutte le \textit{API} esposte, in modo centralizzato.
Grazie al portale, la visualizzazione delle \textit{API} è resa più semplice ed intuitiva, e per ogni servizio è possibile visualizzare la documentazione relativa 
con tutti gli \textit{endpoint} disponibili, i relativi parametri e le risposte attese. Grazie a questa struttura, è possibile che l'utilizzatore del portale possa essere autonomo 
e facilitato dato che non è più necessario conoscere e cercare le \textit{API}, con conseguente risparmio di tempo e risorse.\\
All'interno del portale è inoltre possibile provare i relativi \textit{endpoint} delle singole \textit{API} direttamente nell'applicativo, rendendo il progetto una soluzione completa
in tutti i suoi aspetti che non necessita di applicazione di terze parti per il suo utilizzo, che comunque potranno essere utilizzate grazie alla possibilità di scaricare
lo schema \glsfirstoccur{\gls{yamlg}} di ogni singola \textit{API}.


\section{Obiettivi dello stage}\label{sec:obiettivi-stage}
In questa sezione vengono elencati gli obiettivi prefissati da raggiungere durante lo stage, suddivisi in obbligatori, desiderabili e opzionali.

Si farà riferimento ai requisiti secondo le seguenti notazioni:
\begin{itemize}
    \item \textbf{OB}: per i requisiti obbligatori, vincolanti in quanto obiettivo primario richiesto;
    \item \textbf{DE}: per i requisiti desiderabili, non vincolanti o strettamente necessari, ma dal riconoscibile valore aggiunto;
    \item \textbf{OP}: per i requisiti opzionali, rappresentanti valore aggiunto non strettamente competitivo.
\end{itemize}
Le sigle precedentemente indicate saranno seguite da una coppia sequenziale di numeri, che identificano univocamente ogni requisito.

\subsection*{\emph{Obbligatori}}\label{subsec:obiettivi-obbligatori}
\begin{itemize}
    \item \textbf{OB1}: Realizzazione di un portale che consenta la consultazione degli \glsfirstoccur{\gls{openapig}} dei servizi pubblici e privati offerti da THRON;
    \item \textbf{OB2}: Rendere possibile l'utilizzo delle \textit{API} direttamente dal portale (con inserimento manuale del \textit{token} di autenticazione);
    \item \textbf{OB3}: Documentazione delle funzionalità implementate;
    \item \textbf{OB4}: Realizzazione di test di unità delle funzionalità implementate.
\end{itemize}

\subsection*{\emph{Desiderabili}}\label{subsec:obiettivi-desiderabili}
\begin{itemize}
    \item \textbf{DE1}: Implementare la funzionalità di recupero automatico degli \textit{OpenAPI schemas};
    \item \textbf{DE2}: Implementare la funzionalità di autenticazione al portale.
\end{itemize}

\subsection*{\emph{Opzionali}}\label{subsec:obiettivi-opzionali}
\begin{itemize}
    \item \textbf{OP1}: Implementare la funzionalità di \textit{download} dello schema di uno specifico servizio (formato \textit{YAML});
    \item \textbf{OP2}: Implementare la funzionalità di recupero automatico del \textit{token} di autenticazione per l'utilizzo della \textit{API} direttamente dal portale.
\end{itemize}

\section{Analisi preventiva dei rischi}\label{sec:analisi-rischi}
Durante la fase iniziale sono stati individuati dei possibili rischi a cui si potrà andare incontro. Per contrastare ciò, si è cercato di porre rimedio con delle contromisure appropriate.
\begin{enumerate}
    \item \textbf{\textit{Stack} tecnologico}:\\
        \textbf{Descrizione}: alcune tecnologie utilizzate per lo sviluppo del progetto erano per me nuove o poco conosciute. Ciò poteva portare a un utilizzo scorretto delle tecnologie, non rispettando le \textit{best practice}.\\
        \textbf{Soluzione}: per ovviare a questo rischio, è stato previsto un periodo di formazione iniziale, durante il quale è stato possibile studiare le tecnologie da utilizzare e sperimentarle in piccoli progetti di prova.\\
    \item \textbf{Fattibilità dei requisiti di partenza}:\\
        \textbf{Descrizione}: alcuni dei requisiti risultavano essere complessi da implementare e non era detto che fosse possibile soddisfarli in modo completo con gli strumenti a disposizione.\\
        \textbf{Soluzione}: è stato deciso in accordo con il tutor aziendale di effettuare un'analisi settimanale della situazione al fine di valutare l'andamento del progetto e porre rimedio ad eventuali criticità.\\ 
    \item \textbf{Ritardi nello sviluppo}:\\
        \textbf{Descrizione}: potrebbero verificarsi ritardi nello sviluppo dovuti ad attività esterne al mio progetto (come attività infrastrutturali) necessarie per il suo completamento.\\
        \textbf{Soluzione}: è stato deciso in accordo con il tutor aziendale di effettuare un'analisi settimanale della situazione ed in caso di criticità è stato previsto un dialogo interno con il team infrastruttrale.\\
\end{enumerate}