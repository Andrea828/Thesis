\chapter{Descrizione del progetto di stage}
\label{cap:descrizione-stage}

\intro{Il seguente capitolo vuole introdurre brevemente il progetto affrontato durante lo stage, evidenziando gli obiettivi 
prefissati e i possibili rischi che si potrebbero incontrare.
}

\section{Introduzione al progetto}
Un punto cardine dell'architettura della piattaforma Thron è la suddivisione in micro servizi, che agevolano la manutenibilità e semplificano le operazioni di sviluppo.
Allo stesso tempo rendono però più complessa la consultazione delle API esposte e aumentano la quantità di comunicazione necessaria per mantenere tutti i reparti allineati.
Per questo, tra le necessità che stanno avendo le aree di Prodotto e Revenue (Sales, Marketing e Customer Services) in Thron, è emersa l'esigenza di avere un unico punto
da cui sia possibile consultare in maniera intuitiva, tutte le interfacce delle API che metta a disposizione l'area di Prodotto.\\
Il progetto di stage è consistito nello sviluppo di un portale in Vue.js per favorire la consultazione di tutte le API esposte, in modo centralizzato.
Grazie al portale, la visualizzazione delle API è resa più semplice ed intuitiva, e per ogni servizio è possibile visualizzare la documentazione relativa 
con tutti gli endpoint disponibili, i relativi parametri e le risposte attese. Grazie a questa struttura, è possibile che l'utilizzatore del portale possa essere autonomo 
e facilitato dato che non è più necessario conoscere e cercare le API, con conseguente risparmio di tempo e risorse.\\
All'interno del portale è inoltre possibile provare i relativi endpoint delle singole API direttamente nell'applicativo, rendendo il progetto una soluzione completa
in tutti i suoi aspetti che non necessita di applicazione di terze parti per il suo utilizzo, che comunque potranno essere utilizzate grazie alla possibilità di scaricare
lo schema YAML di ogni singola API.


\section{Obiettivi dello stage}
In questa sezione vengono elencati gli obiettivi prefissati da raggiungere durante lo stage, suddivisi in obbligatori, desiderabili e opzionali.

Si farà riferimento ai requisiti secondo le seguenti notazioni:
\begin{itemize}
    \item \textbf{OB}: per i requisiti obbligatori, vincolanti in quanto obiettivo primario richiesto;
    \item \textbf{DE}: per i requisiti desiderabili, non vincolanti o strettamente necessari, ma dal riconoscibile valore aggiunto;
    \item \textbf{OP}: per i requisiti opzionali, rappresentanti valore aggiunto non strettamente competitivo.
\end{itemize}
Le sigle precedentemente indicate saranno seguite da una coppia sequenziale di numeri, che identificano univocamente ogni requisito.

\subsection*{\emph{Obbligatori}}
\begin{itemize}
    \item \textbf{OB1}: Realizzazione di un portale che consenta la consultazione degli OpenAPIs schemas dei servizi pubblici e privati offerti da Thron;
    \item \textbf{OB2}: Rendere possibile l'utilizzo delle API direttamente dal portale (con inserimento manuale del token di autenticazione);
    \item \textbf{OB3}: Documentazione delle funzionalità implementate;
    \item \textbf{OB4}: Realizzazione di test di unità delle funzionalità implementate.
\end{itemize}

\subsection*{\emph{Desiderabili}}
\begin{itemize}
    \item \textbf{DE1}:  Implementare la funzionalità di recupero automatico degli OpenAPI schemas;
    \item \textbf{DE2}: Implementare la funzionalità di autenticazione al portale.
\end{itemize}

\subsection*{\emph{Opzionali}}
\begin{itemize}
    \item \textbf{OP1}: Implementare la funzionalità di download dello schema di uno specifico servizio (formato YAML);
    \item \textbf{OP2}: Implementare la funzionalità di recupero automatico del token di autenticazione per l'utilizzo della API direttamente dal portale.
\end{itemize}

\section{Analisi preventiva dei rischi}
Durante la fase iniziale sono stati individuati dei possibili rischi a cui si potrà andare incontro. Per contrastare ciò, si è cercato di porre rimedio con delle contromisure appropriate.
% stack tecnologico
\begin{enumerate}
    \item \textbf{Stack tecnologico}:\\
        \textbf{Descrizione}: alcune tecnologie utilizzate per lo sviluppo del progetto erano a me nuove o poco conosciute. Ciò poteva portare a un utilizzo scorretto delle tecnologie, non rispettando le best practice.\\
        \textbf{Soluzione}: per ovviare a questo rischio, è stato previsto un periodo di formazione iniziale, durante il quale è stato possibile studiare le tecnologie da utilizzare e sperimentarle in piccoli progetti di prova.\\
    \item \textbf{Fattibilità dei requisiti di partenza}:\\
        \textbf{Descrizione}: alcuni dei requisiti risultavano essere complessi da implementare e non era detto che fosse possibile soddisfarli in modo completo con gli strumenti a disposizione.\\
        \textbf{Soluzione}: è stato deciso in accordo con il tutor aziendale di effettuare un'analisi settimanale della situazione, per valutare l'andamento del progetto e porre rimedio ad eventuali criticità.\\
    \item \textbf{Ritardi nello sviluppo}:\\
        \textbf{Descrizione}: potrebbero verificarsi ritardi nello sviluppo dovuti ad attività esterne al mio progetto (come attività infrastrutturali) necessarie per il suo completamento.\\
        \textbf{Soluzione}: è stato deciso in accordo con il tutor aziendale di effettuare un'analisi settimanale della situazione ed in caso di criticità è stato previsto un dialogo interno con il team infrastruttrale.\\
\end{enumerate}

% ritardi nello svilippo





% \section{Requisiti e obiettivi}


% \section{Pianificazione}
