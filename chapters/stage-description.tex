\chapter{Descrizione del progetto di stage}
\label{cap:descrizione-stage}

\intro{Il seguente capitolo vuole introdurre brevemente il progetto affrontato durante lo stage}\\

\section{Introduzione al progetto}

\section{Obiettivi dello stage}
In questa sezione vengono elencati gli obiettivi prefissati da raggiungere durante lo stage, suddivisi in obbligatori, desiderabili e opzionali.

Si farà riferimento ai requisiti secondo le seguenti notazioni:
\begin{itemize}
    \item \textbf{OB}: per i requisiti obbligatori, vincolanti in quanto obiettivo primario richiesto;
    \item \textbf{DE}: per i requisiti desiderabili, non vincolanti o strettamente necessari, ma dal riconoscibile valore aggiunto;
    \item \textbf{OP}: per i requisiti opzionali, rappresentanti valore aggiunto non strettamente competitivo.
\end{itemize}
Le sigle precedentemente indicate saranno seguite da una coppia sequenziale di numeri, che identificano univocamente ogni requisito.

\subsection*{\emph{Obbligatori}}
\begin{itemize}
    \item \textbf{OB1}: Realizzazione di un portale che consenta la consultazione degli OpenAPIs schemas dei servizi pubblici e privati offerti da THRON;
    \item \textbf{OB2}: Rendere possibile l'utilizzo delle API direttamente dal portale (con inserimento manuale del token di autenticazione);
    \item \textbf{OB3}: Documentazione delle funzionalità implementate;
    \item \textbf{OB4}: Realizzazione di test di unità delle funzionalità implementate.
\end{itemize}

\subsection*{\emph{Desiderabili}}
\begin{itemize}
    \item \textbf{DE1}:  Implementare la funzionalità di recupero automatico degli OpenAPI schemas;
    \item \textbf{DE2}: Implementare la funzionalità di autenticazione al portale.
\end{itemize}

\subsection*{\emph{Opzionali}}
\begin{itemize}
    \item \textbf{OP1}: Implementare la funzionalità di download dello schema di uno specifico servizio (formato YAML);
    \item \textbf{OP2}: Implementare la funzionalità di recupero automatico del token di autenticazione per l'utilizzo della API direttamente dal portale.
\end{itemize}

\section{Analisi preventiva dei rischi}
Durante la fase iniziale sono stati individuati dei possibili rischi a cui si poteva andare incontro. Per contrastare ciò, si è cercato di porre rimedio con delle contromisure appropriate.
% stack tecnologico
\begin{enumerate}
    \item \textbf{Stack tecnologico}: per ovviare a questo rischio, è stato previsto un periodo di formazione iniziale, durante il quale è stato possibile studiare le tecnologie da utilizzare e sperimentarle in piccoli progetti di prova;
\end{enumerate}

% ritardi nello svilippo





% \section{Requisiti e obiettivi}


% \section{Pianificazione}
