\chapter{Conclusioni}\label{cap:conclusioni}

\intro{
  In questo ultimo capitolo verranno presentate le conclusioni del lavoro svolto, focalizzandosi in particolare sui risultati ottenuti,
  le conoscenze acquisite, le prospettive future e fornendo infine una valutazione personale del lavoro svolto. 
}

\section{Obiettivi raggiunti e consuntivo finale}
Il progetto di stage svolto ha avuto come scopo la creazione di un portale per la consultazione di tutte le \textit{API} messe a disposizione da \textit{THRON}, in un'unica soluzione centralizzata. 
Il portale inoltre doveva permettere ad un'utente di visualizzare i singoli \textit{endpoint} di ogni servizio e fornendo la possibilità di testarli direttamente senza strumenti esterni.
Essendo le \textit{API} informazioni sensibili, il portale per essere navigato in tutte le sue funzionalità, richiedeva un sistema di autenticazione tramite account \textit{Microsoft 365}.
Per la realizzazione del progetto sono state affrontate varie fasi, partendo dall'analisi dei requisiti, dove sono stati definiti i casi d'uso e i requisiti del prodotto.
Successivamente sono state analizzate le tecnologie da utilizzare per la realizzazione del portale, con uno studio di esse, in particolare per quanto riguarda il \textit{framework Vue.js} e il \textit{framework Nest.js}.
È stata poi progettata l'architettura del sistema suguita dall'implementazione in tutte le sue parti.\\

In merito agli obiettivi prefissati per lo stage (in sezione~\ref{sec:obiettivi-stage}), sono stati raggiunti tutti quelli obbligatori e desiderabili, mentre per quanto riguarda gli opzionali, sono stati raggiunti tutti ad eccezzione di uno, ovvero la funzionalità di recupero automatico del token per l'utilizzo delle API direttamente dal portale.
Più nello specifico, sono stati raggiunti i seguenti obiettivi obbligatori:
\begin{itemize}
  \item \textbf{OB1}: Realizzazione di un portale che consenta la consultazione degli OpenAPIs schemas dei servizi pubblici e privati offerti da THRON;
  \item \textbf{OB2}: Rendere possibile l'utilizzo delle API direttamente dal portale (con inserimento manuale del token di autenticazione);
  \item \textbf{OB3}: Documentazione delle funzionalità implementate;
  \item \textbf{OB4}: Realizzazione di test di unità delle funzionalità implementate.
\end{itemize}

Sono stati raggiunti i seguenti obiettivi desiderabili:
\begin{itemize}
  \item \textbf{DE1}:  Implementare la funzionalità di recupero automatico degli OpenAPI schemas;
  \item \textbf{DE2}: Implementare la funzionalità di autenticazione al portale.
\end{itemize}

Infine, sono stati raggiunti i seguenti obiettivi opzionali:
\begin{itemize}
  \item \textbf{OP1}: Implementare la funzionalità di download dello schema di uno specifico servizio (formato YAML).
\end{itemize}

Come accennato in precedenza, non è stato raggiunto l'obiettivo opzionale OP2, ovvero la funzionalità di recupero automatico del token di autenticazione per l'utilizzo della API direttamente dal portale.
Questo obiettivo era presenta nonostante non ci fosse ancora una vera soluzione applicabile, in quanto anche in azienda non era ancora stata trovata una soluzione definitiva.
La problematica principale era che il token di autenticazione per provare le \textit{API} varia a seconda del \textit{client-id} scelto ed ha una durata ridotta e al momento in azienda
non esitono servizi a supporto di ciò.
La soluzione adottata è stata quella di inserire il \textit{token} manualmente, una soluzione che il mio team ha sempre utilizzato fin'ora e che è stata ritenuta più che accettabile.

\section{Conoscenze acquisite}
Il progetto di stage svolto mi ha permesso di acquisire nuove conoscenze e competenze, sia dal punto di vista tecnico che personale, andando a soddisfare le mie aspettative iniziali.
% Per la realizzazione del progetto inoltre sono state fondamentali molte nozioni apprese durante il corso di di laurea triennale.
In primis è stato molto interessante studiare e approfondire i due prodotti aziendali \textit{THRON Pim} e \textit{THRON Dam Platform} di cui non avevo mai sentito parlare prima.
Una delle competenze più significative è stata sicuramente l'approfondimento e l'utilizzo del \textit{framework Vue.js} per la realizzazione di un progetto completo.
Questo perchè al giorno d'oggi, i \textit{framework} per lo sviluppo di applicazioni web sono molto utilizzati e la loro richiesta è in continua crescita.
L'attività di stage mi è stata utile anche per consolidare ulteriormente le mie conoscenze


\section{Scenari di applicabilità e sviluppi futuri}

\section{Valutazione personale}
