\chapter{Conclusioni}\label{cap:conclusioni}

\intro{
  In questo ultimo capitolo verranno presentate le conclusioni del lavoro svolto, focalizzandosi in particolare sui risultati ottenuti,
  le conoscenze acquisite, le prospettive future e fornendo infine una valutazione personale del lavoro svolto. 
}

\section{Obiettivi raggiunti e consuntivo finale}
Il progetto di stage svolto ha avuto come scopo la creazione di un portale per la consultazione di tutte le \textit{API} messe a disposizione da \textit{THRON}, in un'unica soluzione centralizzata. 
Il portale inoltre doveva permettere ad un'utente di visualizzare i singoli \textit{endpoint} di ogni servizio e fornendo la possibilità di testarli direttamente senza strumenti esterni.
Essendo le \textit{API} informazioni sensibili, il portale per essere navigato in tutte le sue funzionalità, richiedeva un sistema di autenticazione tramite account \textit{Microsoft 365}.
Per la realizzazione del progetto sono state affrontate varie fasi, partendo dall'analisi dei requisiti, dove sono stati definiti i casi d'uso e i requisiti del prodotto.
Successivamente sono state analizzate le tecnologie da utilizzare per la realizzazione del portale, con uno studio di esse, in particolare per quanto riguarda il \textit{framework Vue.js} e il \textit{framework Nest.js}.
È stata poi progettata l'architettura del sistema suguita dall'implementazione in tutte le sue parti.\\

In merito agli obiettivi prefissati per lo stage (in sezione~\ref{sec:obiettivi-stage}), sono stati raggiunti tutti quelli obbligatori e desiderabili, mentre per quanto riguarda gli opzionali, sono stati raggiunti tutti ad eccezzione di uno, ovvero la funzionalità di recupero automatico del token per l'utilizzo delle API direttamente dal portale.
Più nello specifico, sono stati raggiunti i seguenti obiettivi obbligatori:
\begin{itemize}
  \item \textbf{OB1}: Realizzazione di un portale che consenta la consultazione degli OpenAPIs schemas dei servizi pubblici e privati offerti da THRON;
  \item \textbf{OB2}: Rendere possibile l'utilizzo delle API direttamente dal portale (con inserimento manuale del token di autenticazione);
  \item \textbf{OB3}: Documentazione delle funzionalità implementate;
  \item \textbf{OB4}: Realizzazione di test di unità delle funzionalità implementate.
\end{itemize}

Sono stati raggiunti i seguenti obiettivi desiderabili:
\begin{itemize}
  \item \textbf{DE1}:  Implementare la funzionalità di recupero automatico degli OpenAPI schemas;
  \item \textbf{DE2}: Implementare la funzionalità di autenticazione al portale.
\end{itemize}

Infine, sono stati raggiunti i seguenti obiettivi opzionali:
\begin{itemize}
  \item \textbf{OP1}: Implementare la funzionalità di download dello schema di uno specifico servizio (formato YAML).
\end{itemize}

Come accennato in precedenza, non è stato raggiunto l'obiettivo opzionale OP2, ovvero la funzionalità di recupero automatico del token di autenticazione per l'utilizzo della API direttamente dal portale.
Questo obiettivo era presenta nonostante non ci fosse ancora una vera soluzione applicabile, in quanto anche in azienda non era ancora stata trovata una soluzione definitiva.
La problematica principale era che il token di autenticazione per provare le \textit{API} varia a seconda del \textit{client-id} scelto ed ha una durata ridotta e al momento in azienda
non esitono servizi a supporto di ciò.
La soluzione adottata è stata quella di inserire il \textit{token} manualmente, una soluzione che il mio team ha sempre utilizzato fin'ora e che è stata ritenuta più che accettabile.

\section{Conoscenze acquisite}
Il progetto di stage svolto mi ha permesso di acquisire nuove conoscenze e competenze, sia dal punto di vista tecnico che personale, andando a soddisfare le mie aspettative iniziali.
% Per la realizzazione del progetto inoltre sono state fondamentali molte nozioni apprese durante il corso di di laurea triennale.
In primis è stato molto interessante studiare e approfondire i due prodotti aziendali \textit{THRON Pim} e \textit{THRON Dam Platform} di cui non avevo mai sentito parlare prima, andando ad avere una visione chiara del contesto aziendale in cui lavoravo.
Una delle competenze più significative è stata sicuramente l'approfondimento e l'utilizzo del \textit{framework Vue.js} per la parte frontend e il \textit{framework Nest.js} per la parte backend del progetto.
Questo perchè al giorno d'oggi, i \textit{framework} per lo sviluppo di applicazioni web sono molto utilizzati e la loro richiesta è in continua crescita. 
E' stato interessante anche imparare il concetto di \textit{middleware} del framework Nest.js e come questo possa essere utilizzato per la gestione delle richieste HTTP.
L'attività di stage mi è stata utile anche per consolidare ulteriormente le mie conoscenze del linguaggio di programmazione \textit{TypeScript}, linguaggio che avevo già utilizzato all'interno del corso di laurea.\\
Ho avuto modo di scoprire nuove nozioni riguardanti l'infrastruttura che supporta un progetto aziendale e tutto ciò che riguarda la configurazione di essa. Questo mi ha permesso di avere una visione
più completa del lavoro che ho svolto e non solo di quello che riguarda lo sviluppo del portale. Infatti ho avuto modo di utilizzare costrutti infrastrutturali che mi hanno agevolato il lavoro, ho dovuto configurare dei file per le \textit{build} e \textit{deploy} del progetto e ho imparato il flusso di lavoro che viene utilizzato in azienda per lo sviluppo di un progetto.\\
L'abilità più importante che ho acquisito è stata sicuramente la capacità di lavorare efficacemente all'interno di un team e poter partecipare attivamente alla maggior parte delle loro attività, a partire da un semplice \textit{daily meeting} fino ad arrivare a una \textit{Sprint Retrospective}.
Riguardo a quest'ultima attività, ho avuto modo di apprendere in modo più approfondito la metodologia Scrum, che avevo già utilizzato all'interno del corso di Ingegneria del Software, ma che non avevo ancora avuto modo di sperimentare in un ambiente lavorativo.

% \section{Scenari di applicabilità e sviluppi futuri}

\section{Valutazione personale}
L'esperienza di stage è stata un capitolo significativo nel mio percorso di crescita personale e professionale.
Ho acquisito conoscenze e competenze tecniche essenziali nel campo dello sviluppo web, che mi saranno fondamentali per il mio futuro.
La possibilità di applicare le nozioni apprese durante il mio percorso di studi in un contesto lavorativo concreto è stata estremamente gratificante.
Inoltre, poter lavorare in stretto contatto con il team di sviluppo mi ha insegnato l'importanza delle dinamiche di gruppo e di come sia fondamentale la comunicazione tra i membri del gruppo per il raggiungimento degli obiettivi prefissati.
Ho sviluppato una maggiore consapevolezza delle mie capacità e delle mie passioni professionali. L'interazione con il team e l'immersione nel mondo dello sviluppo web
mi hanno aiutato a identificare le mie aree di interesse e a capire meglio il mio futuro professionale.\\

I due mesi trascorsi in azienda sono stati molto piacevoli e leggeri, dovuto anche al fatto che l'azienda ha saputo creare un ambiente di lavoro stimolante e giovanile.
Mi ha fatto molto piacere partecipare alle attività ed eventi del team, sentendomi nel mio piccolo parte di esso.
Ho avuto modo di conoscere persone molto competenti e disponibili, che mi hanno aiutato e supportato durante tutto il mio percorso di stage, rendendo il team in cui sono stato inserito un punto chiave per la riuscita del mio stage.
Grazie anche ad un ottimo rapporto creatosi con Andrea, il mio tutor aziendale, ho sempre avuto modo di affrontare le problematiche che si presentavano durante lo stage, trovando sempre una soluzione insieme e permettendomi di raggiungere correttamente gli obiettivi prefissati.\\

In conclusione, ritengo che l'esperienza di stage sia stata per me molto positiva e soddisfacente per il prodotto realizzato, per il percorso fatto per arrivarci, per le conoscenze acquisite e per le persone che ho potuto conoscere.
