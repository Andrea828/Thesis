\chapter{Conclusioni}\label{cap:conclusioni}

\intro{
  In questo ultimo capitolo verranno presentate le conclusioni del lavoro svolto, focalizzandosi in particolare sui risultati ottenuti,
  le conoscenze acquisite e fornendo infine una valutazione personale del lavoro svolto. 
}

\section{Obiettivi raggiunti e consuntivo finale}\label{sec:obiettivi-raggiunti}
Il progetto di stage svolto ha avuto come scopo la creazione di un portale per la consultazione di tutte le \textit{API} messe a disposizione da \textit{THRON}, in un'unica soluzione centralizzata. 
Il portale inoltre doveva permettere ad un utente di visualizzare i singoli \textit{endpoint} di ogni servizio, fornendo la possibilità di provarli direttamente senza l'utilizzo di strumenti esterni.
Essendo le \textit{API} informazioni sensibili, il portale per essere navigato in tutte le sue funzionalità, richiedeva un sistema di autenticazione tramite \textit{account} \textit{Microsoft 365}.
Per la realizzazione del progetto sono state affrontate varie fasi, partendo dall'analisi dei requisiti, dove sono stati definiti i casi d'uso e i requisiti del prodotto.
In seguito, sono state analizzate le tecnologie da impiegare nella creazione del portale, con uno studio di esse, concentrandosi in particolare sul \textit{framework Vue.js} e il \textit{framework Nest.js}.
È stata poi progettata l'architettura del sistema suguita dall'implementazione in tutte le sue parti.\\

In merito agli obiettivi prefissati per lo stage (in sezione~\ref{sec:obiettivi-stage}), sono stati raggiunti tutti quelli obbligatori e desiderabili, mentre per quanto riguarda gli opzionali, sono stati raggiunti tutti ad eccezzione di uno, ovvero la funzionalità di recupero automatico del \textit{token} per l'utilizzo delle \textit{API} direttamente dal portale.
Più nello specifico, sono stati raggiunti i seguenti obiettivi obbligatori:
\begin{itemize}
  \item \textbf{OB1}: Realizzazione di un portale che consenta la consultazione degli \textit{OpenAPIs schemas} dei servizi pubblici e privati offerti da THRON;
  \item \textbf{OB2}: Rendere possibile l'utilizzo delle \textit{API} direttamente dal portale (con inserimento manuale del \textit{token} di autenticazione);
  \item \textbf{OB3}: Documentazione delle funzionalità implementate;
  \item \textbf{OB4}: Realizzazione di test di unità delle funzionalità implementate.
\end{itemize}

Sono stati raggiunti i seguenti obiettivi desiderabili:
\begin{itemize}
  \item \textbf{DE1}:  Implementare la funzionalità di recupero automatico degli \textit{OpenAPI schemas};
  \item \textbf{DE2}: Implementare la funzionalità di autenticazione al portale.
\end{itemize}

Infine, sono stati raggiunti i seguenti obiettivi opzionali:
\begin{itemize}
  \item \textbf{OP1}: Implementare la funzionalità di download dello schema di uno specifico servizio (formato \textit{YAML}).
\end{itemize}

Come accennato in precedenza, non è stato raggiunto l'obiettivo opzionale \textit{OP2}, ovvero la funzionalità di recupero automatico del \textit{token} di autenticazione per l'utilizzo della \textit{API} direttamente dal portale.
Questo obiettivo era presenta nonostante non ci fosse ancora una vera soluzione applicabile, in quanto anche in azienda non è ancora stata trovata una soluzione definitiva.
La problematica principale è che il \textit{token} di autenticazione per testare le \textit{API} varia a seconda del \textit{client-id} scelto e ha una durata limitata, senza servizi aziendali dedicati a questo scopo. La soluzione adottata è stata l'inserimento manuale del \textit{token}, una pratica utilizzata dal mio team finora e considerata più che accettabile.
\section{Conoscenze acquisite}
Il progetto di stage svolto mi ha permesso di acquisire nuove conoscenze e competenze, sia dal punto di vista tecnico che personale, andando a soddisfare le mie aspettative iniziali.\\
Innanzitutto, è stato molto stimolante esplorare in dettaglio i due prodotti aziendali \textit{THRON Pim} e \textit{THRON Dam Platform}, di cui non ero a conoscenza prima. 
Questo mi ha permesso di ottenere una comprensione completa del contesto operativo dell'azienda in cui ho svolto il mio stage.\\
Tra le competenze più significative acquisite, sicuramente spicca l'approfondimento e l'utilizzo del \textit{framework Vue.js} per la parte frontend e del \textit{framework Nest.js} per la parte backend del progetto.
Questi strumenti al giorno d'oggi sono estremamente rilevanti nell'ambito di applicazioni web e la loro richiesta è in costante crescita.
È stato interessante anche imparare il concetto di \textit{middleware} del \textit{framework Nest.js} e come questo possa essere utilizzato per la gestione delle richieste \textit{HTTP}.\\
L'attività di stage mi è stata utile anche per consolidare ulteriormente le mie conoscenze del linguaggio di programmazione \textit{TypeScript}, linguaggio che avevo già utilizzato all'interno del corso di laurea.\\
Ho scoperto nuove nozioni riguardanti l'infrastruttura che supporta un progetto aziendale e tutto ciò che riguarda la configurazione di essa. Questo mi ha permesso di ottenere una visione
più completa del lavoro svolto, andando oltre l'attività dello sviluppo del portale.
Ho imparato e utilizzato strumenti infrastrutturali che hanno semplificato il mio lavoro, come i costrutti, e ho configurato file per la \textit{build} e il \textit{deploy} del prodotto comprendendo il flusso di lavoro utilizzato in azienda per lo sviluppo di progetti.\\
La competenza più importante acquisita durante l'esperienza di stage è stata la capacità di lavorare in modo efficace in un team e poter partecipare attivamente a tutte le attività, dalle riunioni quotidiane fino alle riunioni di \textit{Sprint Retrospective}.
Riguardo a quest'ultima attività, ho approfondito la metodologia \textit{Scrum}, che avevo già utilizzato all'interno del corso di Ingegneria del Software, ma che non avevo ancora sperimentato in un ambiente lavorativo.

\section{Valutazione personale}\label{sec:valutazione-personale}

L'esperienza di stage rappresenta un capitolo fondamentale nel mio percorso di crescita personale e professionale. Durante questo periodo, ho avuto l'opportunità di acquisire conoscenze e competenze tecniche 
fondamentali nel campo dello sviluppo web, che saranno un pilastro essenziale per il mio futuro professionale. La possibilità di mettere in pratica le nozioni apprese durante il mio percorso di studi in un contesto
lavorativo concreto è stato molto gratificante e ha contribuito in modo significativo al mio apprendimento.\\
In aggiunta, lavorare a stretto contatto con il team di sviluppo mi ha insegnato l'importanza delle dinamiche di gruppo e della comunicazione efficace tra i suoi membri per il raggiungimento degli obiettivi prefissati.
% Ho sviluppato una maggiore consapevolezza delle mie capacità e delle mie passioni professionali grazie all'interazione con il team durante lo stage. Questa esperienza mi hanno aiutato a identificare le mie aree di interesse e a definire 
% maggiormente la mia futura carriera professionale.\\
Durante lo stage, ho acquisito una maggiore consapevolezza delle mie capacità e delle mie passioni professionali grazie all'interazione con il team, il che mi ha aiutato a identificare le mie aree di interesse e a delineare con maggior chiarezza la mia futura carriera professionale.\\
I due mesi trascorsi in azienda sono stati estremamente piacevoli, un fatto che va attribuito in parte all'azienda stessa, che ha creato un ambiente di lavoro stimolante e giovane.
È stato un piacere partecipare alle attività ed eventi del team, sentendomi parte integrante di questo gruppo. Ho avuto l'opportunià di conoscere persone competenti e disponibili, che mi hanno sostenuto e guidato 
lungo tutto il mio percorso in azienda. Questo ha reso il mio team di lavoro un elemento chiave per il successo del mio stage.\\
Un ringraziamento particolare va ad Andrea, il mio tutor aziendale, con il quale ho instaurato un ottimo rapporto. La sua disponibilità e la sua competenza hanno sempre garantito che non affrontassi complicazioni bloccanti da solo, 
permettendomi di raggiungere con successo gli obiettivi prestabiliti.\\

In conclusione, valuto l'esperienza di stage in maniera estremamente positiva e gratificante. Sono soddisfatto del risultato che ho ottenuto, del cammino che ho percorso, delle competenze 
che ho acquisito e delle persone che ho avuto l'opportunità di conoscere. Questa esperienza ha arricchito il mio percorso professionale e sono profondamente grato per questa opportunità, 
determinato a sfruttarla al massimo nelle sfide future che mi attendono.