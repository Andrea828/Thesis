\chapter{Introduzione}
\label{cap:introduzione}

% Introduzione al contesto applicativo.\\

% \noindent Esempio di utilizzo di un termine nel glossario \\
% \gls{api}. \\

% \noindent Esempio di citazione in linea \\
% \cite{site:agile-manifesto}. \\

% \noindent Esempio di citazione nel pie' di pagina \\
% citazione\footcite{womak:lean-thinking} \\
\intro{Il seguente capitolo vuole introdurre brevemente l'azienda e il relativo contesto aziendale.}\\


\section{L'azienda}

THRON S.p.A. è un'azienda italiana  con sede a Piazzola sul Brenta specializzata nello sviluppo di SaaS (Software as a Service) e opera nel settore dei DAM (Digital Asset Management), offrendo servizi di marketing e business intelligence. \\
Il suo prodotto principale è la "Thron DAM Platform", una piattaforma per la gestione dei contenuti digitali, nata con l'obiettivo di valorizzare e gestire le informazioni sui prodotti in modo separato dalla piattaforma di distribuzione finale. \\
La Thron DAM Platform è una soluzione completa per la gestione dei contenuti aziendali, inclusi documenti, video, immagini, audio e molto altro. L'obiettivo è garantire una gestione centralizzata dei contenuti e semplificarne l'adattamento e 
la distribuzione su diversi canali in modo efficiente. La piattaforma è supportata da un motore semantico che consente l'arricchimento dei contenuti con tag e metadati, facilitando l'organizzazione e la categorizzazione dei materiali. \\
THRON S.p.A. ha strutturato l'area R\&D in due principali team: il team Contenuti, focalizzato sulle tematiche legate al DAM e alle sue funzionalità, e il team Prodotto, responsabile della gestione del PIM (Product Information Management) 
e delle funzionalità legate alle tag sui prodotti. Il metodo di lavoro adottato è il framework agile SCRUM, che coinvolge attivamente gli stakeholder nel processo decisionale, incoraggiando suggerimenti e miglioramenti. \\
L'azienda mira a una vasta ed eterogenea clientela, per cui offre un prodotto flessibile e adattabile alle specifiche esigenze dei clienti. Inizialmente, la piattaforma viene venduta con un modulo base che offre le funzionalità standard, ma attraverso un marketplace, è 
possibile acquistare o aggiungere moduli gratuiti che ampliano e migliorano la piattaforma. \\
La cultura di prossimità con gli stakeholder e il contatto costante con diverse esigenze dei clienti portano THRON a un continuo processo di innovazione per soddisfare le richieste e rimanere all'avanguardia nel settore. \\
Attualmente, THRON conta circa cinquanta dipendenti, suddivisi in team in base alle rispettive competenze. La collaborazione tra i dipartimenti e la capacità di adattarsi alle esigenze del mercato e dei clienti sono elementi fondamentali
 nel successo dell'azienda. Un'ulteriore peculiarità dell'azienda è la possibilità di offrire soluzioni personalizzate per specifici casi d'uso dei clienti, garantendo un servizio su misura per le loro esigenze. \\
La continua evoluzione e innovazione della Thron DAM Platform, insieme all'impegno costante nel fornire soluzioni di alto livello, consolidano la posizione di THRON S.p.A. nel mercato del Digital Asset Management, 
portando il "made in Italy" in un ambito altamente competitivo e in continua crescita. \\

\section{Metodologie di sviluppo}

Introduzione all'idea dello stage.

\section{Strumenti di sviluppo}
code build e code commit 

\section{Organizzazione del testo}

% \begin{description}
    
%     \item[{\hyperref[cap:descrizione-stage]{Il terzo capitolo}}] approfondisce ...
    
%     \item[{\hyperref[cap:analisi-requisiti]{Il quarto capitolo}}] approfondisce ...
    
%     \item[{\hyperref[cap:progettazione-codifica]{Il quinto capitolo}}] approfondisce ...
    
%     \item[{\hyperref[cap:verifica-validazione]{Il sesto capitolo}}] approfondisce ...
    
%     \item[{\hyperref[cap:conclusioni]{Nel settimo capitolo}}] descrive ...
% \end{description}

% Riguardo la stesura del testo, relativamente al documento sono state adottate le seguenti convenzioni tipografiche:
% \begin{itemize}
% 	\item gli acronimi, le abbreviazioni e i termini ambigui o di uso non comune menzionati vengono definiti nel glossario, situato alla fine del presente documento;
% 	\item per la prima occorrenza dei termini riportati nel glossario viene utilizzata la seguente nomenclatura: \emph{parola}\glsfirstoccur;
% 	\item i termini in lingua straniera o facenti parti del gergo tecnico sono evidenziati con il carattere \emph{corsivo}.
% \end{itemize}
