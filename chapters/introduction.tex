\chapter{Introduzione}
\label{cap:introduzione}

% Introduzione al contesto applicativo.\\

% \noindent Esempio di utilizzo di un termine nel glossario \\
% \gls{api}. \\

% \noindent Esempio di citazione in linea \\
% \cite{site:agile-manifesto}. \\

% \noindent Esempio di citazione nel pie' di pagina \\
% citazione\footcite{womak:lean-thinking} \\
\intro{Il seguente capitolo vuole introdurre brevemente l'azienda e il relativo contesto aziendale.}\\


\section{L'azienda}

THRON S.p.A è un'azineda italiana con sede a Piazzola sul Brenta specializzata nello sviluppo di SaaS (Software as a Service).
I suoi prodotti principali sono la Thron DAM Platform e il Thron PIM. 
Il primo è una piattaforma per la gestione dei contenuti digitali, nata con l'obiettivo di valorizzare e gestire le informazioni 
sui prodotti in modo separato dalla piattaforma di distribuzione finale.
D'altro canto, il Thron Pim è una soluzione per la gestione delle informazioni sui prodotti, che si concentra
sulla raccolta, organizzazione e distribuzione delle informazioni relative ai prodotti di un'azienda.
In sintesi, il primo prodotto si concetra sulla gestione dei contenuti digitali, il secondo invece
si concentra sulla gestione delle informazioni sui prodotti.
L'obiettivo è garantire una gestione centralizzata dei contenuti e semplificarne l'adattamento e la distribuzione su diversi
canali in modo efficiente.
All'interno dell'azienda, l'area R\&D è suddivisa in due team: il team Prodotto, responsabile della gestione
del PIM, e il team Contenuti, focalizzato sulle tematiche legate al DAM.\\
Durante il mio stage presso l'azienda, sono stato inserito come sviluppatore frontend all'interno dell'area Prodotto.

\section{Metodologie di sviluppo}


\section{Strumenti di sviluppo}
Strumenti
\begin{itemize}
  \item \textbf{Teams}
  \item \textbf{Microsoft 365}
  \item \textbf{Jira}
  \item \textbf{AWS}
  \item \textbf{Fork}
  \item \textbf{Confluence}
  \item \textbf{StarUML}
\end{itemize}

\section{Organizzazione del testo}

\begin{description}
    
    \item[{\hyperref[cap:descrizione-stage]{Il primo capitolo}}] descrive il progetto svolto durante il periodo di stage, mettendo in risalto gli obiettivi imposti dall'azienda.
    
    \item[{\hyperref[cap:analisi-requisiti]{Il secondo capitolo}}] descrive l'analisi dei requisiti del progetto di stage, evidenziando i casi d'uso.
    \item     
    \item[{\hyperref[cap:progettazione-codifica]{Il terzo capitolo}}] descrive 
    
    \item[{\hyperref[cap:verifica-validazione]{Il quarto capitolo}}] descrive 
    
    \item[{\hyperref[cap:conclusioni]{Il quinto capitolo}}] descrive
\end{description}

% Riguardo la stesura del testo, relativamente al documento sono state adottate le seguenti convenzioni tipografiche:
% \begin{itemize}
% 	\item gli acronimi, le abbreviazioni e i termini ambigui o di uso non comune menzionati vengono definiti nel glossario, situato alla fine del presente documento;
% 	\item per la prima occorrenza dei termini riportati nel glossario viene utilizzata la seguente nomenclatura: \emph{parola}\glsfirstoccur;
% 	\item i termini in lingua straniera o facenti parti del gergo tecnico sono evidenziati con il carattere \emph{corsivo}.
% \end{itemize}
